\documentclass[12pt, twoside]{report}
\usepackage[a4paper, margin=3.3cm]{geometry}
\usepackage{url,amsfonts,epsfig}
\usepackage[italian, english]{babel}
\usepackage{ragged2e}
\usepackage{blindtext}

\usepackage[T1]{fontenc}
\usepackage[utf8]{inputenc}

\usepackage{float}
\usepackage{placeins}

\usepackage{lipsum}
\usepackage{fancyhdr}
\usepackage{graphicx}
\usepackage{tikz}
\usepackage{csquotes}
\usepackage{float}
\usepackage{listings}
\usepackage{xcolor}
\graphicspath{ {./images/} }
\usepackage[hyperindex]{hyperref} %per l'indice interattivo
\usepackage{caption}
\usepackage{subcaption}
\usepackage{chngcntr} % Per gestire la numerazione
\hypersetup{
    colorlinks=true,
    linkcolor=black,
    filecolor=black,      
    urlcolor=black,
    bookmarks=true,
    pdfpagemode=FullScreen,
    citecolor=black
}
\usepackage[square,numbers]{natbib}
\usepackage{amsmath}
\usepackage{amssymb}
\usepackage{enumitem}
\usepackage{svg}
\usepackage{subcaption}
\usepackage{tikz}
\usepackage{pgfplots}
\usepackage{minted}
\usepackage{fvextra}

\setminted{
    linenos,                           % Mostra i numeri di riga
    frame=lines,                       % Aggiunge una cornice intorno al codice
    bgcolor=lightgray!20,              % Sfondo grigio chiaro
    fontsize=\footnotesize,            % Font leggermente più piccolo per il codice
    breaklines=true,                   % Spezza automaticamente le righe lunghe
    breakautoindent=true,              % Mantiene l'indentazione quando si spezza una riga
    breaksymbol=\textcolor{red}{\tiny\ensuremath{\hookrightarrow}} % Simbolo per indicare spezzature
}

\renewcommand{\listoflistingscaption}{Elenco dei codici}
% Ridefinisci il nome del contatore per i listing
\renewcommand{\listingscaption}{Codice}

% Numerazione in base al capitolo
\counterwithin{listing}{chapter} % Numerazione relativa ai capitoli

% Definizione del formato della numerazione
\renewcommand{\thelisting}{\thechapter.\arabic{listing}}

\definecolor{codegreen}{rgb}{0,0.6,0}
\definecolor{codegray}{rgb}{0.5,0.5,0.5}
\definecolor{codepurple}{rgb}{0.58,0,0.82}
\definecolor{backcolour}{rgb}{0.95,0.95,0.92}

\lstdefinestyle{mystyle}{
    backgroundcolor=\color{backcolour},   
    commentstyle=\color{codegreen},
    keywordstyle=\color{magenta},
    numberstyle=\tiny\color{codegray},
    stringstyle=\color{codepurple},
    basicstyle=\ttfamily\fontsize{10}{12}\selectfont,
    breakatwhitespace=false,         
    breaklines=true,                 
    captionpos=b,                    
    keepspaces=true,                 
    numbers=left,                    
    numbersep=5pt,                  
    showspaces=false,                
    showstringspaces=false,
    showtabs=false,                  
    tabsize=2
}
\renewcommand\lstlistingname{Codice} % Cambia "Listing" in "Codice"
\lstset{style=mystyle}
%\fancyhead[LE, RO]{\thepage}
%%\fancyhead[RE]{\leftmark}
%%\fancyhead[LO]{\leftmark}
%\fancyfoot[LE, RO]{\thepage}
%\fancyfoot[CO, CE]{}

% --- FRONTE-RETRO ---
%\fancyhead[LE]{\textit{\leftmark}}    % Titolo capitolo a sinistra nelle pagine pari in corsivo
%\fancyhead[RO]{\textit{\leftmark}}    % Titolo capitolo a destra nelle pagine dispari in corsivo
%\fancyhead[LO]{\thepage}              % Numero pagina a sinistra nelle pagine dispari
%\fancyhead[RE]{\thepage}              % Numero pagina a destra nelle pagine pari

%\fancyfoot[LE,RO]{\empty}             % Nessun contenuto al centro del piè di pagina
%\fancyfoot[LO,RE]{\thepage}           % Numero pagina a sinistra nelle pagine dispari e a destra nelle pari
%\fancyfoot[CO,CE]{\empty}             % Nessun contenuto al centro del piè di pagina
% --- FRONTE-RETRO ---

\renewcommand{\headrulewidth}{0pt}       % Rimuove la barra dell'intestazione
\fancyhf{}                           % Disabilita tutte le intestazioni e i piè di pagina

\fancyfoot[LE,RO]{\empty}             % Nessun contenuto ai lati del piè di pagina
\fancyfoot[LO,RE]{\empty}             % Nessun contenuto ai lati del piè di pagina
\fancyfoot[CO,CE]{\thepage}           % Numero pagina centrato

% STEP 1: include the package
\usepackage{frontespizio}
% STEP 2: Front-page configuration
\Universita{Università degli Studi di Napoli ``Parthenope''}
\Facolta{Scuola interdipartimentale delle Scienze, dell'Ingegneria e della Salute}
\Dipartimento{Dipartimento di Scienze e Tecnologie}
\CorsoDiLaurea{Corso di Laurea Triennale in Informatica}
\AnnoAccademico{Anno Accademico 2025/2026}
\Titolo{IDENTIFICAZIONE E TRACCIAMENTO DELL'EVOLUZIONE DI FLARE SOLARE}
\Relatore{\textsc{Prof. Emanuel Di Nardo}}
\RelatoreLabel{\textsc{Relatore}} %optional, default: Relatore
%\relandcorrelsep{2em} %optional, vertical space between relatori and correlatori default: 1.5ex
\Correlatore{
\textsc{Prof. Angelo Ciaramella}}
\CorrelatoreLabel{\textsc{Correlatore}} %optional, default: Correlatore
%\Correlatore{Dott.Foo \textsc{Bar}} % can add as many "Correlatori" as you wish
\Candidato{\textsc{Sara Terlizzi}} %only one candidate is currently supported
\CandidatoLabel{\textsc{Candidata}} %optional, default: Candidato
\Matricola{\textsc{0124002161}}
\Logo{Figs/University_Crest} %path to logo image
\LogoWidth{6cm} %optional, default: 3cm
\LogoPosition{below-uni} % or top, or below-title, or above-title, or no-logo

% \setlength{\parindent}{0pt} % rimuove l'indentazione globalmente su tutta la  tesi 

\begin{document}
    % STEP 3: use \makefrontpage and/or \makefrontpagealt
    \begin{titlepage}
    
    \pagestyle{empty}
    \makefrontpage
    
    \newpage
\thispagestyle{empty}
\null
%\newpage
%\thispagestyle{empty}
%\null
%\newpage
%\thispagestyle{empty}
%\null
%\newpage
\selectlanguage{english}
\begin{abstract}
Abstract italiano

\begin{center}
    \textbf{Abstract (English)}
\end{center}
Abstract inglese

\end{abstract}

\selectlanguage{italian}
    
    \end{titlepage}
    \pagestyle{empty}
    %\fontsize{12}{14}\selectfont

    \selectlanguage{italian}

    \pagenumbering{gobble} % Disabilita la numerazione delle pagine

    %dedica
    \begin{flushright}
    \textit{A chi non respira più con me,\\
    ma continua a vivere nei miei ricordi e nel mio cuore:\\
    A Nonna ed Emidio, che non smetteranno mai di mancarmi.
    \vskip 1.0cm
    Ai miei Genitori, all'idea che per la prima volta \\
    sono riuscita a ripagare -per davvero- ogni vostro sforzo.\\
    A mio fratello Francesco, ti ho desiderato con tutto il mio cuore ed \\
    -ora che sei qui- voglio che tu sia la versione migliore di me.\\
    Alla mia famiglia tutta, dagli incoraggiamenti prima di ogni prova \\
    alle telefonate post-esame.
    \vskip 1.0cm
    Ad ogni persona che ho incontrato sulla mia strada\\
    -che sia rimasta o che sia stata solo di passaggio-\\
    siete stati essenziali per srotolare il groviglio che c'è in me.
    \vskip 1.0cm
    A chiunque abbia creduto in me,\\
    nelle mie potenzialità e nel mio modo di essere,\\
    anche quando -io stessa- non ero in grado di farlo.
    \vskip 1.0cm
    A chi lotta contro il nemico invisibile dell'ansia di non essere abbastanza.\\
    Alle notti insonni e alle lacrime.\\
    Alla paura che ti mangia dentro e sembra non esistere via di fuga.\\
    Al buio che mi ha fatto apprezzare la luce.\\
    Oggi ho vinto io.
    }
    \end{flushright}
            
    %citazione
    \newpage
    \emph{\vskip 6cm
    “Puoi sprecare la tua vita a tracciare confini oppure puoi decidere di vivere superandoli. Alcuni sono molto difficili da superare. Però una cosa la so: se sei pronto a correre il rischio, la vita dall'altra parte è spettacolare.”}
    \vskip 0.5cm
    \begin{figure}[hb]
    \centering
    \includegraphics[width=0.6\linewidth]{Figs/notte_stellata.jpg}
    \end{figure}


    
    \tableofcontents
    \listoffigures
    \listoftables
    \listoflistings
    \cleardoublepage
    \pagestyle{fancy}
    \pagenumbering{arabic} % Inizia la numerazione delle pagine in numeri arabi
    \cleardoublepage
    \chapter{Introduzione}
\label{cap:1}
Nel presente capitolo, si andrà a spiegare il contesto scientifico in cui si inserisce il lavoro di tesi, illustrando le motivazioni alla base della scelta del tema affrontato e definendo l'obiettivo principale della ricerca. In particolare, l'attenzione è rivolta allo studio delle regioni attive solari, in quanto sedi privilegiate dei fenomeni energetici più intensi osservati sul Sole.\\
Il capitolo fornisce inoltre una panoramica delle implicazioni scientifiche e applicative legate all'analisi di tali regioni e anticipa la struttura generale della tesi.\\

\section{Contesto e Motivazione}
Il Sole è da sempre oggetto di numerosi studi nell'ambito dell'astrofisica, in quanto sorgente di fenomeni magnetici ed energetici che influenzano l'ambiente spaziale circostante. Tra tali fenomeni, i \textbf{flare solari} rappresentano eventi improvvisi ed altamente energetici, caratterizzati dal rilascio di grandi quantità di energia sottoforma di radiazione elettromagnetica. \\
Numerose osservazioni hanno evidenziato come tali eventi abbiano origine principalmente nelle \textbf{regioni attive solari}, aree localizzate della fotosfera caratterizzate da intensi e complessi campi magnetici, spesso associate alla presenza di macchie solari. La complessità strutturale e l'evoluzione temporale delle regioni attive svolgono un ruolo fondamentale nell'origine dei flare solari e dei fenomeni ad essi associati, quali le espulsioni di massa coronale (\textit{Coronal Mass Ejections, CME}) e le particelle energetiche solari (\textit{Solar Energetic Particles, SEP}).\\
Quando tali eventi si propagano verso la Terra, possono dare origine a \textbf{tempeste geomagnetiche} e a una vasta gamma di effetti sull'ambiente spaziale circumterrestre, con conseguenze negative sulle infrastrutture tecnologiche \cite{camporeale2018machine}. Tra le manifestazioni più rilevanti vi è la generazione di \textbf{correnti geomagneticamente indotte} (\textit{Geomagnetically Induced Currents, GIC}) nelle reti elettriche terrestri, che possono causare danni ai trasformatori e interruzioni della fornitura di energia, come dimostrato da eventi storici quali il \textit{blackout} della rete elettrica di Hydro-Québec del marzo 1989 \cite{bolduc2002gic}, durante il quale l'intero sistema passò da condizioni operative nominali allo spegnimento completo in circa 92 secondi, lasciando oltre sei milioni di persone senza elettricità in una giornata particolarmente fredda. Ulteriori esempi sono rappresentati dai numerosi guasti ai trasformatori che ridussero drasticamente la capacità della rete elettrica sudafricana durante le intense tempeste geomagnetiche dell'ottobre 2003 \cite{gaunt2007transformer}.\\
Le emissioni in raggi X e nell'ultravioletto estremo (\textit{Extreme Ultraviolet, EUV}) prodotte dai flare solari più intensi possono modificare la struttura dell'ionosfera sul lato diurno della Terra, compromettendo la propagazione dei segnali radio e influenzando il funzionamento dei sistemi globali di navigazione satellitare (\textit{Global Navigation Satellite Systems, GNSS}) \cite{hapgood2010ionospheric}. Tali sistemi, oltre alla navigazione, forniscono servizi di temporizzazione ad alta precisione, fondamentali per settori critici come quello finanziario. Ulteriori effetti includono un aumento del \textit{drag} atmosferico sui satelliti in orbita bassa, che può determinare una progressiva perdita di quota e, nei casi più estremi, il rientro prematuro in atmosfera. Un esempio recente di questo fenomeno è rappresentato dalla perdita di una parte della flotta di satelliti Starlink in seguito a un episodio di intensa attività geomagnetica.\\
Sia le espulsioni di massa coronale sia i flare solari sono inoltre associati a \textbf{tempeste di radiazione solare}, ovvero improvvisi aumenti del flusso di particelle energetiche (protoni, particelle alfa e ioni più pesanti), che possono determinare un significativo incremento dell'ambiente radiativo nello spazio circumterrestre e, in alcuni casi, anche all'interno dell'atmosfera terrestre, fino a raggiungere il livello del mare. Le tempeste di radiazione hanno effetti su una vasta gamma di sistemi elettrici ed elettronici e possono rappresentare un rischio radiativo, seppur limitato, per l'uomo, sia nello spazio sia a bordo di aeromobili. Tali effetti risultano particolarmente rilevanti per l'aviazione, soprattutto considerando il ruolo sempre più centrale dei dispositivi digitali nei sistemi di controllo del volo, nonché nei sistemi di navigazione e comunicazione.\\
Alla luce di ciò, l'identificazione e il tracciamento temporale delle regioni attive solari risultano di fondamentale importanza per comprendere e, potenzialmente, prevedere l'attività solare.
\begin{figure}[H]
  \centering
  \includegraphics[width=0.4\textwidth]{Figs/Cap1/flare_solare.jpg}
  \caption[Flare Solare]{Immagine del Sole in cui sono in atto flare solari}
\end{figure}

\section{Obiettivo della Tesi}
Da quanto illustrato finora, lo studio dell'attività solare riveste un ruolo fondamentale nell'analisi dello \textit{space weather}, ovvero l'insieme dei fenomeni astrofisici che possono influenzare le tecnologie spaziali e terrestri.\\
Poiché gli eventi più rilevanti in questo contesto hanno origine prevalentemente nelle regioni attive del Sole, la presente tesi si propone di:
\begin{enumerate}
    \item Sviluppare un sistema automatico per l'identificazione delle regioni attive solari attraverso l'analisi dei magnetogrammi del disco solare, prodotti dallo strumento \textbf{Helioseismic and Magnetic Imager (HMI)} a bordo del satellite \textbf{Solar Dynamics Observatory (SDO)}.
    \item Utilizzare il \textit{data product} \textbf{HARP} (\textit{Helioseismic and Magnetic Imager Active Region Patches}), fornito dal team scientifico del Solar Dynamics Observatory, che offre una rappresentazione strutturata delle aree magneticamente attive sulla superficie solare, permettendo l'annotazione dei magnetogrammi e lo sviluppo di un metodo di localizzazione automatica delle regioni attive, basato sull'estrazione delle \textit{bounding box}.
    \item Riadattare il modello di \textit{object detection} \textbf{YOLOv7}, rendendolo utilizzabile in un contesto astrofisico e integrandolo con la libreria \textbf{Norfair} per il \textit{tracking} temporale delle regioni attive.
    \item Utilizzare un dataset annotato per l’addestramento del modello YOLOv7 (opportunamente modificato), con l’obiettivo di riconoscere automaticamente le regioni attive solari.
\end{enumerate}
\textbf{YOLOv7} è un modello di \textit{object detection} preesistente, ideato per identificare e localizzare oggetti in immagini RGB in modo accurato ed efficiente \cite{wang2022yolov7}. Per adattarlo all'analisi dei magnetogrammi solari e delle regioni attive descritte nei dati HARP, sono state apportate le seguenti modifiche:
\begin{itemize}
  \item caricamento di immagini in formato scientifico (file \texttt{.h5} contenenti magnetogrammi e \textit{bounding box} HARP);
  \item estrazione delle \textit{bounding box} a partire dai dati HARP;
  \item addestramento del modello per la rilevazione automatica delle regioni attive in immagini solari;
  \item integrazione con la libreria Norfair per il \textit{tracking} temporale delle regioni attive rilevate.
\end{itemize}
Tali modifiche consentono di sfruttare al meglio YOLOv7 in un contesto scientifico diverso da quello per cui è stato originariamente concepito, ma mantenendo al contempo l'efficienza e la rapidità del modello originale.\\
I dati utilizzati in questo lavoro sono pubblicamente accessibili dal Joint Science Operations Center (JSOC), la struttura responsabile dell'elaborazione e della distribuzione dei dati del Solar Dynamics Observatory (SDO), gestita dalla Stanford University in collaborazione con la NASA \cite{pesnell2012solar}.\\
La selezione dei dati è avvenuta a cura di Elizabeth Doria Rosales, dottoranda in fisica presso l'Università di Trento.

\section{Struttura della Tesi}
La tesi è strutturata nei seguenti capitoli:
\begin{itemize}
    \item \hyperref[cap:2]{\textbf{Capitolo 2}}: Panoramica delle tecnologie e metodologie utilizzate, con focus sul modello YOLOv7 e sulla libreria Norfair.
    \item \hyperref[cap:3]{\textbf{Capitolo 3}}: Illustrazione delle soluzioni tecniche adottate.
    \item \hyperref[cap:4]{\textbf{Capitolo 4}}: Analisi dei risultati sperimentali, con valutazioni.
    \item \hyperref[cap:5]{\textbf{Capitolo 5}}: Conclusioni ed eventuali sviluppi futuri.
\end{itemize}
Infine, le conclusioni racchiudono una riflessione sui risultati ottenuti, sui limiti del lavoro effettuato e le potenziali direzioni da poter intraprendere in futuro.

    \chapter{Modelli e Metodologie Utilizzate}
\label{cap:2}
Nel presente capitolo sono descritte le principali tecnologie e metodologie adottate durante lo sviluppo del progetto. La parte introduttiva fornisce una spiegazione concettuale dei concetti legati all'Intelligenza Artificiale, con focus specifico sui concetti di Object Detection e Bounding Box, essenziali per comprendere la natura del problema trattato. Si passa, poi, alla descrizione del linguaggio di programmazione e delle librerie di supporto, motivandone la scelta; particolare rilevanza è data a strumenti come Miniconda, Pytorch e CUDA, di fondamentale importanza nello sviluppo del lavoro di tesi. Il nucleo centrale del capitolo è dedicato all'architettura YOLOv7, con il suo funzionamento e le sue innovazioni rispetto ai modelli precedenti, caratteristiche che l'hanno reso la soluzione più adeguata alla sfida proposta. Infine, è descritta la libreria Norfair, utilizzata per il tracciamento, con il suo funzionamento e le sue caratteristiche.

\section{Panoramica sull'Intelligenza Artificiale, l'Object Detection e le Bounding Box}
Nel contesto del progetto di tesi, queste tecnologie sono state fondamentali: l'\textbf{Intelligenza Artificiale} ha permesso l'addestramento di un modello per un compito non usuale; l'\textit{Object Detection} è stata la metodologia per localizzare le regioni attive nei magnetogrammi; le \textit{Bounding Box} hanno rappresentato lo strumento pratico per delimitarle visivamente.

\subsection{Intelligenza Artificiale}
L'\textbf{Intelligenza Artificiale} (IA) è un campo dell'informatica che si propone di sviluppare sistemi in grado di svolgere attività tipicamente umane, quali ragionamento, apprendimento automatico, elaborazione del linguaggio naturale e percezione visiva\cite{russell2010artificial}.
Negli ultimi anni, l'IA ha compiuto enormi passi in avanti, e questo lo deve all'evoluzione del \textit{machine learning} (ML), e in particolare del \textit{deep learning} (DL). Tutto ciò ha reso possibile affrontare con successo problemi complessi come il riconoscimento facciale, la traduzione automatica, il rilevamento di oggetti e persino la guida autonoma \cite{goodfellow2016deep}. \\
Nel contesto dell’IA moderna, le tecniche più rilevanti sono:
\begin{itemize}
    \item \textbf{Machine Learning:} metodi che apprendono da dati osservati senza essere esplicitamente programmati per ogni compito.
    \item \textbf{Deep Learning:} architetture neurali multilivello, capaci di apprendere rappresentazioni gerarchiche e complesse dei dati.
    \item \textbf{Apprendimento supervisionato e non supervisionato:} rispettivamente con o senza etichette nei dati di input.
\end{itemize}

\subsubsection{Machine Learning}
Il \textbf{Machine Learning} (ML) è una branca dell’intelligenza artificiale che si occupa di progettare algoritmi in grado di apprendere automaticamente dai dati, migliorando le proprie prestazioni nel tempo senza essere esplicitamente programmati \cite{mitchell1997machine}.
Un algoritmo di Machine Learning è addestrato su un insieme di dati \textit{training set}, per poi essere valutato su dati non visti, detti \textit{test set}, così da poter verificarne la sua capacità di generalizzazione. \\
Le principali modalità di apprendimento nel ML sono:
\begin{itemize}
    \item \textbf{Apprendimento supervisionato:} l’algoritmo apprende da dati etichettati, ovvero ogni esempio presente dataset è associato ad un output già noto. Questa rappresenta la modalità più usata, soprattutto per quanto riguarda la classificazione di immagini o il riconoscimento vocale.
    \item \textbf{Apprendimento non supervisionato:} i dati non hanno etichette, per cui è l’algoritmo stesso a tentare di identificare strutture nascoste o raggruppamenti nei dati. Questa modalità è usata nel \textit{clustering}, nella compressione o nel rilevamento di anomalie.
    \item \textbf{Apprendimento per rinforzo:} un agente interagisce con un ambiente e apprende tramite un sistema di ricompense e penalità, ottimizzando le proprie decisioni. Questa modalità è impiegata, ad esempio, nei videogiochi o nella robotica autonoma.
\end{itemize}

\subsubsection{Deep Learning}
Il \textbf{Deep Learning} (DL), come accennato prima, è una sottoclasse del ML che utilizza \textit{reti neurali profonde}, composte da molti strati (\textit{layers}) nascosti. Tali modelli sono particolarmente efficienti nel trattare dati complessi e non strutturati, come immagini, audio e testo \cite{goodfellow2016deep}.
Grazie alla grande disponibilità di dati e potenza computazionale, il Deep Learning ha rivoluzionato il campo della \textit{computer vision}, rendendo possibili compiti prima irrisolvibili, tra cui:
\begin{itemize}
    \item Riconoscimento facciale in tempo reale
    \item Traduzione automatica basata sul contesto
    \item Rilevamento e classificazione di oggetti in immagini ad alta risoluzione
\end{itemize}

\subsection{Object Detection}
L'\textbf{Object Detection} è un'area fondamentale della \textit{computer vision} che ha l'obiettivo di localizzare e classificare automaticamente uno o più oggetti in un'immagine o in un video \cite{zhao2019object}.
I modelli di Object Detection producono come output sia la categoria di ciascun oggetto (es. ``persona'', ``auto'', ``segnale stradale''), sia una bounding box che ne racchiude l'area nell'immagine. \\
Le principali tecniche si dividono in due grandi famiglie:
\begin{itemize}
    \item \textbf{Two-stage detectors:} suddividono il processo in due fasi distinte: dapprima vi è la generazione di regioni proposte (\textit{Region Proposal Network, RPN}) e successivamente la classificazione delle regioni. Esempi noti sono R-CNN, Fast R-CNN e Faster R-CNN \cite{ren2015faster}.
    \item \textbf{One-stage detectors:} eseguono direttamente la classificazione e la regressione delle bounding box in un'unica fase, abbreviando notevolmente i tempi. I principali rappresentanti di questa famiglia sono YOLO (\textit{You Only Look Once}) - di cui è stato fatto uso per il presente progetto di tesi - \cite{redmon2016you} e SSD (\textit{Single Shot MultiBox Detector}) \cite{liu2016ssd}.
\end{itemize}

\subsubsection{Metriche di Valutazione}
Per valutare quantitativamente le prestazioni, è necessario stabilire un criterio oggettivo per classificare la correttezza delle predizioni. A tal fine si utilizza l'\textbf{Intersection over Union (IoU)}, che misura il grado di sovrapposizione geometrica tra la bounding box predetta dal modello e quella reale (ground truth), calcolata come il rapporto tra l'area dell'intersezione e l'area dell'unione delle due box:
  \[
  \text{IoU} = \frac{\text{Area of Overlap}}{\text{Area of Union}}
  \] \\
Fissata una soglia di tolleranza $\alpha$ (tipicamente $0.5$), ogni predizione viene classificata in una delle seguenti categorie:
\begin{itemize}
  \item \textbf{True Positive (TP)}: rilevamento corretto, caratterizzato da un $\text{IoU} \ge \alpha$.
  \item \textbf{False Positive (FP)}: rilevamento errato o "falso allarme", in cui il modello predice un oggetto dove non esiste o con una sovrapposizione insufficiente ($\text{IoU} < \alpha$).
  \item \textbf{False Negative (FN)}: mancato rilevamento, ovvero un oggetto reale presente nel dataset che il modello non è riuscito a identificare.
\end{itemize}
Sulla base di queste quantità, si definiscono le metriche derivate:
\begin{itemize}
  \item \textbf{Precision ($P = \frac{TP}{TP + FP}$)}: indica l'affidabilità del modello, esprimendo la percentuale di predizioni corrette rispetto al totale dei rilevamenti effettuati.
  \item \textbf{Recall ($R = \frac{TP}{TP + FN}$)}: misura la capacità di copertura del modello, indicando la frazione di oggetti reali correttamente individuati.
  \item \textbf{mAP (mean Average Precision)}: rappresenta la metrica globale di sintesi. Essa è calcolata come la media delle Average Precision (AP) di tutte le classi, dove l'AP corrisponde all'area sottesa alla curva Precision-Recall, offrendo una valutazione che bilancia sia la precisione che la recall al variare della soglia di confidenza.
\end{itemize}

\subsection{Bounding Box}
La \textbf{bounding box} è un rettangolo utilizzato per racchiudere un oggetto rilevato all'interno di un'immagine. È solitamente rappresentata da quattro coordinate:
\[
(x_{\text{min}}, y_{\text{min}}, x_{\text{max}}, y_{\text{max}})
\]
dove:
\begin{itemize}
    \item $(x_{\text{min}}, y_{\text{min}})$ rappresenta il vertice in alto a sinistra,
    \item $(x_{\text{max}}, y_{\text{max}})$ rappresenta il vertice in basso a destra.
\end{itemize}
Alternativamente, può essere espressa con centro e dimensioni:
\[
(x_{\text{center}}, y_{\text{center}}, w, h)
\]
dove $w$ e $h$ sono larghezza e altezza.\\
L’accuratezza delle bounding box è cruciale nei sistemi \textit{real-time}, dove la minima imprecisione può compromettere l’affidabilità dell'intero sistema.

\section{Python}
\textbf{Python} è un linguaggio di programmazione ad alto livello, ampiamente utilizzato per la sua semplicità di sintassi, versatilità e la presenza di un ecosistema ricco di librerie.
Spicca per i suoi punti di forza: adattabilità ad ogni piattaforma, interattività e dinamicità e protipizzazione rapida.\\
Nel contesto di questo progetto di tesi, Python è stato scelto come linguaggio principale per lo sviluppo degli script e l'esecuzione degli elaborazione dati, visione artificiale e deep learning. Si è rivelato essenziale per integrare le diverse componenti del sistema: dalla lettura dei dati scientifici HARP, alla pre-elaborazione delle immagini, fino all'addestramento ed inferenza del modello.

\subsection{Librerie ausiliarie}
\begin{itemize}
  \item \textbf{NumPy} \cite{numpy}: è la libreria fondamentale per il calcolo scientifico in Python. Fornisce strutture dati per la gestione di array multidimensionali e innumerevoli funzioni matematiche per operazioni vettoriali e matriciali. Nel progetto, è stata utilizzata per la manipolazione e preparazione dei dati, ovvero dei magnetogrammi, rappresentati come array multidimensionali, e per gestire le coordinate delle bounding box durante le fasi di addestramento e validazione.
  \item \textbf{SciPy} \cite{scipy}: estende le funzionalità di NumPy, offrendo strumenti per operazioni matematiche avanzate come integrazione numerica, ottimizzazione, interpolazione e algebra lineare. Nel progetto, è stata utilizzata per supportare calcoli complessi durante l’analisi dei dati e l’elaborazione dei risultati, in particolare ha fornito strumenti di supporto per analisi statistiche preliminari sulle proprietà magnetiche delle regioni attive descritte nei dati HARP.
  \item \textbf{OpenCV} \cite{opencv}: libreria open source per la computer vision e l’elaborazione delle immagini. Offre algoritmi ottimizzati per operazioni di filtraggio, rilevamento di caratteristiche, trasformazioni geometriche e manipolazione delle immagini. Nel progetto, è stata usata in primis perché viene utilizzata dal modello, ma anche per la pre-elaborazione e manipolazione dei magnetogrammi solari come, ad esempio, la normalizzazione dei valori e la conversione dei formati, rendendoli idonei per l'input da passare al modello.
  \item \textbf{Matplotlib} \cite{matplotlib}: libreria di visualizzazione dati che permette la creazione di grafici statici, animati e interattivi. Consente di rappresentare chiaramente e in modo personalizzabile i risultati ottenuti, facilitando l’analisi e la documentazione visiva dei dati elaborati. Nel progetto, è stata la libreria indispensabile per visualizzare i risultati, ad esempio per disegnare le bounding box predette direttamente sui magnetogrammi solari e per generare grafici che mostrano l'andamento delle performance del modello.
\end{itemize}

\section{Miniconda}
\textbf{Miniconda} è una distribuzione minimale di Conda, un sistema di gestione degli ambienti e dei pacchetti Python. È stato utilizzato per creare ambienti isolati e riproducibili, permettendo di configurare ciascuno di essi con una propria versione di Python e delle librerie di terze parti necessarie (come PyTorch, NumPy, OpenCV).\\
Questa gestione consente di mantenere sotto controllo le dipendenze e garantisce che le specifiche versioni delle librerie siano compatibili fra loro, evitando conflitti che avrebbero potuto compromettere l'addestramento del modello. \cite{miniconda}

\section{PyTorch}
\textbf{PyTorch} è un framework open-source per il deep learning sviluppato da Meta AI. Esso fornisce un'interfaccia dinamica e intuitiva per la definizione e l'ottimizzazione delle reti neurali, ed è particolarmente apprezzato per il suo supporto nativo all’accelerazione tramite GPU \cite{paszke2019pytorch}.\\
In questo progetto di tesi, è stato il framework su cui si basa l'intero processo di addestramento ed inferenza del modello, offrendo la flessibilità desiderata e necessaria per apportare le modifiche atte all'adattamento del compito scientifico.

\section{CUDA}
\textbf{CUDA} (\textit{Compute Unified Device Architecture}) è una piattaforma di calcolo parallelo sviluppata da NVIDIA, che consente l’utilizzo delle GPU per compiti generici di calcolo.\\
Data l'enorme mole di dati e la complessità del modello utilizzato, l'accelerazione hardware è stata indispensabile per ridurre i tempi sia di addestramento che di inferenza, rendendo possibile l'esecuzione di più esperimenti in tempi alquanto ragionevoli.\\
Nel contesto di questa tesi, la corretta configurazione di CUDA e la sua compatibilità con la versione di PyTorch si sono rivelate essenziali per il garantire il funzionamento ottimale dei modelli su GPU.\cite{cuda}.

\section{YOLOv7}
\textbf{YOLOv7} (\textit{You Only Look Once version 7}) è uno dei modelli più recenti e avanzati per il rilevamento oggetti in tempo reale, appartenente alla famiglia di algoritmi YOLO. Sviluppato per ottenere alte prestazioni sia in termini di accuratezza che di velocità, si distingue per una serie di ottimizzazioni che lo rendono adatto ad innumerevoli applicazioni, anche su dispositivi con risorse computazionali limitate \cite{wang2022yolov7}. \\
La scelta di tale modello è motivata dall'ottimo compromesso tra efficienza e precisione, una caratteristica cruciale nell'analisi di grandi moli di dati astrofisici.

\subsection{Funzionamento e Caratteristiche}
YOLOv7 è progettato per realizzare una \textit{pipeline end-to-end}, che riceve in input un’immagine e restituisce in output le bounding box e le classi associate agli oggetti rilevati. \\
Il funzionamento, durante la fase di inferenza, si basa sulla suddivisione dell'immagine in griglie e sull'applicazione di convoluzioni profonde per produrre:
\begin{itemize}
  \item Le coordinate delle bounding box.
  \item La classe predetta per ogni oggetto rilevato.
  \item La probabilità (\textit{confidenza}) associata a ciascuna predizione.
\end{itemize}
Si tratta del cuore del sistema di rilevamento sviluppato nel presente lavoro di tesi, necessario per il successivo tracciamento.\\
L'architettura separa nettamente la configurazione degli iperparametri e degli \textit{anchor} dai pesi pre-addestrati e offre diverse caratteristiche avanzate:
\begin{itemize}
  \item \textbf{Efficienza}: elevata precisione nel rilevamento pur mantenendo alte velocità di inferenza.
  \item \textbf{Versatilità}: supporto per \textit{Object Detection}, \textit{Instance Segmentation} e \textit{Pose Estimation}.
  \item \textbf{Scalabilità}: disponibilità di architetture derivate (ad esempio YOLOv7-tiny, YOLOv7-X, YOLOv7-W6) adattabili a diverse situazioni e capacità di calcolo.
  \item \textbf{Ottimizzazione Hardware}: supporto nativo per GPU CUDA, che consente un'inferenza efficiente su hardware NVIDIA.
  \item \textbf{Multitasking}: possibilità di usare \textit{multi-head} per compiti multitask (ad esempio detection + keypoints).
\end{itemize}
In fase di addestramento, inoltre, il modello utilizza tecniche avanzate di ottimizzazione che migliorano l’apprendimento, senza aumentare la complessità dell’inferenza.

\subsection{Avanzamenti nei Detector in Tempo Reale}
YOLOv7 rappresenta un significativo passo avanti nel campo del \textit{real-time object detection}. L’architettura introduce il concetto di \textbf{Trainable Bag-of-Freebies}: un insieme di moduli e strategie ottimizzate che migliorano significativamente la fase di addestramento senza incrementare il costo computazionale durante l'inferenza. \\
Tutti i modelli YOLO sono stati addestrati da zero sul dataset COCO, senza utilizzare pesi pre-addestrati o dati esterni \cite{wang2022yolov7}.

\subsubsection{E-ELAN: Extended Efficient Layer Aggregation Networks}
Per migliorare l’apprendimento delle rappresentazioni senza interrompere il percorsi di propagazione del gradiente, YOLOv7 introduce una nuova \textit{backbone} denominata \textbf{E-ELAN}, visibile in Figura \ref{fig:architettura_eelan} \\
A differenza delle architetture tradizionali, E-ELAN mantiene fissa la struttura di transizione, ma agisce sui blocchi computazionali utilizzando:
\begin{itemize}
  \item \textbf{Group Convolution}: per suddividere i canali in gruppi.
  \item \textbf{Shuffle \& Merge Cardinality}: per mescolare e fondere le \textit{feature map} provenienti dai diversi gruppi.
\end{itemize}
\begin{figure}[tb]
  \centering
  \includegraphics[width=1.0\textwidth]{Figs/Cap2/e-elan.png}
  \caption[Architettura E-ELAN]{L’architettura E-ELAN mantiene invariato il percorso di propagazione del gradiente dell’architettura originale, ma utilizza convoluzioni di gruppo per aumentare la cardinalità delle feature aggiunte. Le feature provenienti da diversi gruppi vengono combinate attraverso operazioni di mescolamento e fusione, migliorando così la varietà delle rappresentazioni apprese e l’efficienza nell’uso dei parametri e del calcolo.}
  \label{fig:architettura_eelan}
\end{figure}

\subsubsection{Compound Scaling}
Il ridimensionamento dei modelli avviene, solitamente, modificando profondità, larghezza o risoluzione.\\
Tuttavia, nei modelli basati su concatenazione (come YOLO), scalare solo la profondità causa una variazione indesiderata dei canali di input nei layer successivi, rompendo l'equilibrio computazionale (Figura \ref{fig:scaling}).\\
Per risolvere questo problema, YOLOv7 propone il \textbf{Compound Scaling}, una tecnica che scala profondità e larghezza in modo congiunto: quando si scala la profondità di un blocco computazionale, si scala simultaneamente la larghezza dei layer di transizione.\\
Le relazioni sono definite come:
\[
\begin{aligned}
d' &= d \cdot \alpha \\
w' &= w \cdot \beta
\end{aligned}
\]
dove \(\alpha\) e \(\beta\) sono fattori empiricamente scelti per mantenere l'architettura ottimale al variare della scala del modello.
\begin{figure}[tb]
  \centering
  \includegraphics[width=1.0\textwidth]{Figs/Cap2/compound-scaling.png}
  \caption[Compound Scaling]{Comportamento e la soluzione proposta per lo scaling nei modelli basati su concatenazione, composta da tre sottoparti: (a): Dimostra che scalare solo la profondità (depth) in un blocco concatenativo fa aumentare la larghezza del layer di uscita; (b): Mostra che l’output più largo influenza la transizione successiva, creando problemi strutturali; (c): Presenta la soluzione proposta: uno scaling combinato (compound scaling), in cui si scala la profondità nel blocco e la larghezza nei layer di transizione per mantenere la coerenza architetturale.}
  \label{fig:scaling}
\end{figure}

\subsubsection{Planned Re-parameterized Convolution}
YOLOv7 ottimizza le convoluzioni ri-parametrizzate (\textit{RepConv}), che combinano diverse operazioni in un unico layer per l'inferenza.\\
Gli autori hanno osservato che la connessione identità -tipica delle RepConv standard- degrada le prestazioni, se applicata indiscriminatamente su connessioni residue o concatenate. La soluzione adottata è, pertanto, la \textbf{Planned RepConv}, che utilizza una variante priva di identità (\textit{RepConvN}) nei layer sensibili, garantendo la compatibilità strutturale con architetture come ResNet o DenseNet.

\subsubsection{Label Assignment: Coarse-to-Fine Lead Guided}
Una delle innovazioni più rilevanti riguarda la strategia di assegnazione delle etichette (\textbf{label assignment}), durante l'addestramento, con supervisione profonda (\textit{deep supervision}), visibile in Figura \ref{fig:assegnazione_etichette}.\\
Il modello utilizza due "teste" (\textit{heads}):
\begin{itemize}
  \item \textbf{Lead Head}: responsabile dell'output finale, genera predizioni raffinate.
  \item \textbf{Auxiliary Head}: assiste l'addestramento nei livelli intermedi, ricevendo etichette guidate dal lead head, non direttamente dal ground truth.
\end{itemize}
La strategia \textbf{Coarse-to-Fine} prevede che il Lead Head guidi l'apprendimento dell'Auxiliary Head:
\begin{enumerate}
  \item Il Lead Head genera etichette di alta precisione per se stesso.
  \item Dalle predizioni del Lead Head vengono derivate etichette grezze per l'Auxiliary Head. Per evitare che quest'ultima faccia \textit{overfitting}, viene applicato un vincolo superiore (\textit{Upper Bound Constraint}) che limita l'assegnazione delle etichette in base alla distanza dal centro dell'oggetto:
    \[
    \text{Score}_{\text{coarse}} = \max \left( 0, 1 - \frac{\text{dist}}{\text{thresh}} \right)
    \]
\end{enumerate}
Questo permette alla testa ausiliaria di apprendere meglio le informazioni contestuali, migliorando la capacità di generalizzazione del modello (\textit{recall}), mentre la testa principale si focalizza sulla precisione.
\begin{figure}[tb]
  \centering
  \includegraphics[width=1.0\textwidth]{Figs/Cap2/coarse-fine-labels.png}
  \caption[Strategia di Label Assignment]{Assegnazione di etichette coarse per l’head ausiliario e fine per il lead head. Rispetto al modello standard (a), lo schema (b) include una testa ausiliaria. Diversamente dall’assegnazione indipendente delle etichette (c), sono proposte due nuove strategie: (d) assegnazione guidata dal lead head e (e) assegnazione guidata coarse-to-fine. Quest’ultima genera contemporaneamente le etichette per il training del lead head e della testa ausiliaria combinando le predizioni del lead head con il ground truth.}
  \label{fig:assegnazione_etichette}
\end{figure}

\subsubsection{Altri Trainable-of-Freebies}
Oltre alle innovazioni architetturali, YOLOv7 integra una serie di "trucchi" di addestramento (\textit{Bag-of-Freebies}) ottimizzati per massimizzare le prestazioni senza costi aggiuntivi:
\begin{itemize}
    \item \textbf{Batch Normalization fusa}: durante l'inferenza, i layer di Batch Normalization vengono fusi con i layer convoluzionali, semplificando la struttura della rete e riducendo la latenza.
    \item \textbf{Implicit Knowledge}: un concetto derivato da YOLOR, qui semplificato tramite l'uso di vettori statici pre-calcolati che vengono combinati con le feature map.
    \item \textbf{EMA Model (Exponential Moving Average)}: utilizzo di una media mobile esponenziale dei pesi del modello durante il training per ottenere il modello finale di test, tecnica che aumenta la robustezza e la stabilità delle predizioni.
\end{itemize}

\subsubsection{Confronto delle Prestazioni}
Le innovazioni introdotte permettono a YOLOv7 di superare i modelli precedenti. Come evidenziato nella Tabella \ref{tab:yolo_comparison}, a parità di risoluzione, YOLOv7 ottiene risultati migliori in mAP, mantenendo un \textit{framerate} (FPS) più elevato rispetto a YOLOR e YOLOv5.
\begin{table}[tb]
\centering
\begin{tabular}{lcccc}
\hline
Modello & FPS & mAP & Parametri & FLOPs \\
\hline
YOLOv5-X & 83 & 50.7\% & 86.7M & 205.7G \\
YOLOR-CSP-X & 87 & 52.7\% & 96.9M & 226.8G \\
\textbf{YOLOv7-X} & \textbf{114} & \textbf{52.9\%} & 71.3M & 189.9G \\
\hline
\end{tabular}
\caption[Confronto prestazioni YOLOv7]{Confronto tra YOLOv7 ed altri modelli stato dell'arte (640×640).}
\label{tab:yolo_comparison}
\end{table}
La Tabella \ref{tab:scaling_comparison} dimostra, inoltre, l'efficacia del Compound Scaling: rispetto allo scaling tradizionale (solo larghezza o solo profondità), il metodo combinato ottiene il miglior risultato in accuratezza con un incremento minimo di calcoli.
\begin{table}[tb]
\centering
\begin{tabular}{lccc}
\hline
Modello & mAP & Parametri & FLOPs \\
\hline
Base & 51.7\% & 47.0M & 125.5G \\
Solo larghezza & 52.4\% & 73.4M & 195.5G \\
Solo profondità & 52.7\% & 69.3M & 187.6G \\
\textbf{Compound (YOLOv7-X)} & \textbf{52.9\%} & 71.3M & 189.9G \\
\hline
\end{tabular}
\caption[Efficienza del Compound Scaling]{Confronto dell'efficienza del Compound Scaling e metodi di scaling tradizionali.}
\label{tab:scaling_comparison}
\end{table}

\section{Norfair}
\textbf{Norfair} è una libreria open source sviluppata in Python per il \textit{tracking} multi-oggetto in tempo reale \cite{norfair}. Essa è progettata per integrarsi modularmente con qualsiasi sistema di rilevamento che fornisca coordinate spaziali (ad esempio le coordinate \((x, y)\) dei centri delle bounding box). \\
La libreria si occupa esclusivamente della parte di tracking, ovvero dell'associazione temporale dei rilevamenti (\textit{detections}) frame per frame, mantenendo un identificatore univoco stabile per ogni oggetto monitorato nel video o nel flusso di immagini (esempio in Figura \ref{fig:esempio_tracking}). Non includendo la componente di rilevamento, essa offre la flessibilità di utilizzare qualsiasi \textit{detector} esterno (come YOLO, Detectron2, MediaPipe).\\
Nel progetto sviluppato, svolge il ruolo chiave di "inseguire" le regioni attive nel tempo: una volta che YOLOv7 le ha identificate nel singolo magnetogramma, Norfair associa le rilevazioni tra frame consecutivi, dando la possibilità di studiare l'evoluzione di una specifica regione solare. 

\subsection{Funzionamento e Caratteristiche}
Il funzionamento di Norfair si basa sui seguenti concetti chiave:
\begin{itemize}
  \item \textbf{Detections}: ogni oggetto rilevato in un fotogramma è rappresentato da coordinate spaziali, tipicamente il centro della bounding box o altri punti di interesse.
  \item \textbf{Tracks}: sono le tracce temporali che mantengono lo storico delle posizioni e degli identificatori degli oggetti nel tempo.
  \item \textbf{Funzione di distanza}: per associare i nuovi rilevamenti alle tracce esistenti, viene utilizzata una funzione di distanza personalizzabile (ad esempio euclidea o basata su caratteristiche visive).
  \item \textbf{Assegnazione}: ogni nuova detection viene assegnata alla traccia più vicina purché la distanza sia inferiore a una soglia predefinita (\textit{distance threshold}); in caso contrario, viene creata una nuova traccia.
\end{itemize}
Ad ogni nuovo fotogramma, Norfair riceve l’elenco delle \textit{detections}, aggiorna le tracce esistenti o ne crea di nuove e rimuove quelle non aggiornate per un certo numero di frame, garantendo così un tracking coerente anche in presenza di occlusioni temporanee o movimenti rapidi.
\begin{figure}[tb]
  \centering
  \includegraphics[width=1\textwidth]{Figs/Cap2/traffic.jpg}
  \caption{Esempio di multi-tracking di Norfair}
  \label{fig:esempio_tracking}
\end{figure}




    \chapter{Applicazione realizzata}
\label{cap:3}
Nel presente capitolo, sarà descritta dettagliatamente l'intera architettura software del sistema sviluppato. L'analisi si concentrerà sulle scelte implementative che hanno permesso di trasformare un framework di Object Detection generico in uno strumento utilizzabile in presenza di dati astrofisici. In particolare, saranno illustrate le modifiche apportate al codice sorgente di YOLOv7 per renderlo compatibile con il formato dei magnetogrammi, assieme al modo in cui sono state integrate le librerie ausiliarie per realizzare effettivamente il processo di predizione.

\section{Adattamento del modello YOLOv7}
Per poter utilizzare un modello preesistente come YOLOv7 nel dominio dell'astrofisica solare, il primo passo necessario è stato quello di creare un metodo che permettesse al framework di leggere e di interpretare un formato di dati per cui non è stato progettato. Quindi, in questa sezione, saranno esplicitate tutte le modifiche tecniche apportate a livello pratico al codice sorgente, in modo da superare questa limitazione e rendere il modello nella sua interezza compatibile con i dati scientifici HARP.
\subsection{Implementazione di un Data Loader per Dati Scientifici}
Come ampiamente anticipato, YOLOv7 è progettato per operare su un dataset di immagini tradizionali (es. \texttt{JPEG, PNG}), le cui annotazioni sono fornite in semplici formati di testo, in cui ogni riga contiene classe e coordinate normalizzate della bounding box. I dati HARP, invece, presentano una complessità strutturale che ne impedisce l'utilizzo diretto, data da:
\begin{itemize}
    \item \textbf{Formato contenitore:} i dati sono racchiusi da un formato \texttt{HDF5} (\texttt{.h5}), ovvero un file system gerarchico pensato per dati scientifici. All'interno del singolo file, i dati sono suddivisi in gruppi distinti e non solo come una semplice immagine.
    \item \textbf{Annotazioni implicite:} le coordinate delle bounding box non sono fornite esplicitamente, bensì è necessaria l'estrazione, partendo dagli attributi incastonati nella gerarchia del file.
\end{itemize}
Per superare l'incompatibilità tra le parti, è stato implementato \textit{ex novo} lo script\\
\texttt{utils/dataset\_h5.py}, il quale -non solo- agisce come \textit{traduttore specializzato} tra il formato specifico e l'input standard, ma implementa due strategie software avanzate per garantire maggiore efficienza e robustezza: un meccanismo di \textit{caching su disco} per accelerare gli avvii e una \textit{gestione delle eccezioni} per ignorare i file corrotti in fase di addestramento. Tali compiti sono eseguiti dalla \textbf{classe DatasetH5}, all'interno della quale, la logica è implementata nei due metodi principali:
\begin{itemize}
    \item \textbf{Metodo Costruttore (\textit{Caching Intelligente per l'Efficienza})}\\
    Il metodo \texttt{\_\_init\_\_} gestisce la pre-elaborazione di tutti i metadati del dataset. Per evitare l'onerosa ripetizione di tale operazione, che può richiedere anche ore, ad ogni avvio del training, è stata implementata una logica di \textit{caching}. I passi sono i seguenti:
    \begin{enumerate}
        \item \textbf{Verifica della Cache}: Come prima operazione, lo script controlla l'esistenza dei file \texttt{\_labels.npy} e \texttt{\_shapes.npy} nella cartella di appartenenza; essi sono progettati per contenere -rispettivamente- tutte le etichette elaborate e le dimensioni originali dell'immagine.
        \item \textbf{Caricamento Veloce (\textit{Cache Hit})}: Se i suddetti file esistono (pre-elaborazione già eseguita in passato), le informazioni sono caricate immediatamente in memoria tramite la funzione \texttt{np.load()}, riducendo il tempo di avvio da ore a pochi secondi: questo rende il processo di sperimentazione molto più agile.
        \item \textbf{Creazione della Cache (\textit{Cache Miss})}: Se i suddetti file non esistono (solitamente solo alla prima esecuzione), è avviato il processo di analisi completo. Si itera su ogni singolo file \texttt{.h5} e sono eseguite una serie di operazioni:
        \begin{enumerate}
            \item \textbf{Accesso e Navigazione}: è aperta la struttura interna del file e letta la matrice del magnetogramma dal percorso \texttt{/magnetogram/data}, memorizzando le dimensioni originali;
            \item \textbf{Parsing e Validazione dei Metadati}: lo script naviga fino al percorso \texttt{/harp/metadata} e itera su ogni sottogruppo, ciascuno dei quali rappresenta una regione attiva. Per ognuna di esse, è eseguito un procedimento di validazione per mantenere l'integrità dell'etichetta:
            \begin{enumerate}
                \item \textbf{Verifica di Completezza}: si controlla la presenza di tutti gli attributi necessari per la definizione delle bounding box: \texttt{CRPIX1} e \texttt{CRPIX2} (coordinate x e y del pixel di riferimento della regione), \texttt{CRSIZE1} e \texttt{CRSIZE2} (larghezza e altezza in pixel della regione).
                Se anche solo uno di questi attributi manca, la regione è scartata.
                \item \textbf{Verifica delle Dimensioni}: si controlla che i valori di \texttt{CRSIZE1} (larghezza) e \texttt{CRSIZE2} (altezza) siano strettamente maggiori di zero. Ovviamente, etichette con dimensioni nulle o negative sono considerate corrotte e, quindi, ignorate.
                \item \textbf{Verifica dei Limiti (\textit{Boundary Check})}: dopo aver calcolato le coordinate normalizzate del centro della bounding box, si verifica che queste rientrino effettivamente nei limti dell'immagine (comprese tra 0.0 e 1.0). Le etichette il cui centro cade al di fuori dei bordi vengono scartate: questo garantisce che solo le regioni con un centro visibile siano considerate valide, pur ammettendo quelle che potrebbero estendersi parzialmente oltre il bordo.
            \end{enumerate} 
            \item \textbf{Calcolo e Memorizzazione}: Solo una volta superati i controlli di validazione, le coordinate di una regione sono calcolate, trasformandole da un tipo \textit{"punto di origine + dimensioni"} ad un sistema del tipo \textit{"centro + dimensioni"} e memorizzate in una lista temporanea.
        \end{enumerate}
        \item \textbf{Salvataggio della Cache}: Al termine di questo lungo processo, le liste contenenti tutte le etichette e le dimensioni vengono salvate su disco nei rispettivi file \texttt{.npy}. In questo modo, la cache è creata e resterà disponibile per eventuali avvii futuri.
    \end{enumerate}
    \item \textbf{Metodo di Accesso ai Dati (\textit{Caricamento Robusto})}\\
    Il metodo \texttt{\_\_getitem\_\_} è chiamato ripetutamente durante il training da processi paralleli (\texttt{workers}) per caricare un singolo magnetogramma e prepararlo per il modello. La sua logica è resa robusta grazie al blocco \texttt{try...except} per garantire la stabilità dell'addestramento. All'interno del suddetto metodo, sono eseguite una serie di operazioni:
    \begin{enumerate}
        \item \textbf{Lettura del Dato}: Viene aperto il file \texttt{.h5} richiesto e ne legge i dati relativi al magnetogramma. Poichè questa è l'operazione più a rischio in caso di file corrotti, viene effettuato il \texttt{try...except}.
        \item \textbf{Pre-processing}: Se la lettura ha successo, il magnetogramma è sottoposto ad una serie di trasformazioni, in modo da renderlo un input valido ed efficace per la rete neurale:
        \begin{enumerate}
            \item \textbf{Pulizia dei Valori Anormali}: I dati scientifici possono contenere valori non validi come \texttt{Nan} (Not A Number) o \texttt{inf} (infinito), spesso causati da errori di misurazione o di calcolo. Non essendo valori numericamente stabili, se passati ad una rete neurale, causerebbero un fallimento nel processo di training. Per questo, è eseguita una pulizia preventiva tramite la funzione \texttt{np.nan\_to\_num}, che sostituisce le eventuali occorrenze di tali valori con il valore neutro \texttt{(0.0)}.
            \item \textbf{Clipping dei Valori}: I magnetogrammi presentano una notevole variazione di intensità tra le diverse aree, ma i valori più significativi per l'identificazione delle regioni attive si trovano in un intervallo specifico, mentre, invece, valori oltremodo alti/bassi rappresentano -nella maggior parte dei casi- rumori. Per questo motivo è necessaria un'operazione di \textit{clipping} mediante la funzione \texttt{np.clip()} della libreria \texttt{NumPy}: essa "taglia" i valori del magnetogramma, forzando tutti i pixel al di fuori a rientrare nell'intervallo predefinito, aiutando, in questo modo, il modello a concentrarsi sulle caratteristiche più significative.
            \item \textbf{Normalizzazione Min-Max}: Poichè le reti neurali apprendono più efficientemente quando i dati in input sono scalati in un intervallo piccolo, si applica una normalizzazione che riscala tutti i valori dei pixel all'interno dell'intervallo \texttt{[0,1]}, garantendo la stessa scala di valori che -in questo modo- accelera la convergenza del training.
            \item \textbf{Ridimensionamento}: YOLOv7 richiede che la dimensione di input sia 640x640, quindi, i magnetogrammi sono ridimensionati mediante la funzione \texttt{cv2.resize()}, mediante l'utilizzo di un'interpolazione lineare che rappresenta un ottimo compromesso tra qualità visiva e velocità di calcolo.
            \item \textbf{Conversione a 3 Canali}: Dato che YOLOv7 accetta unicamente immagini a tre canali, è stato necessario rendere i magnetogrammi (a singolo canale) compatibili, per cui il canale unico è stato duplicato tre volte mediante la funzione \texttt{np.stack()}. Ciò significa che non viene aggiunta informazione, ma c'è un semplice riadattamento del dato all'input richiesto dal modello. Infine, avviene la conversione in un tensore PyTorch nel formato \texttt{[Canali, Altezza, Larghezza]}.
        \end{enumerate}
        \item \textbf{Gestione dell'Errore}: Se una qualsiasi operazione sul file fallisce, viene cattura l'eccezione. Piuttosto che interrompere l'intero training, lo script stampa a video un avviso con il nome del file problematico e restituisce un tensore  di zeri (corrispondente ad un'immagine nera).
    \end{enumerate}
La struttura così descritta rende l'intero processo efficiente in termini di tempo e robusto in caso di errori nel dataset. Il tutto è stato fondamentale per l'intero progetto: i dati scientifici sono stati utilizzati come se fossero immagini.\\
Di seguito, è riportato il codice sorgente dello script.\\

\begin{lstlisting}[language=Python,caption={[dataset\_h5.py]Implementazione commentata della classe \texttt{DatasetH5}. Ogni riga di codice e' accompagnata da una spiegazione per illustrare in dettaglio il processo di caricamento, validazione e pre-elaborazione dei dati dei magnetogrammi.},label={lst:dataset_h5_commentato}]
# File: dataset_h5.py

# --- BLOCCO IMPORTAZIONI ---
import torch  # Libreria principale per il deep learning.
from torch.utils.data import Dataset  # Classe base di PyTorch per creare dataset personalizzati.
import h5py  # Libreria specifica per leggere file in formato HDF5.
import numpy as np  # Libreria per il calcolo numerico, usata per manipolare gli array di dati.
import cv2  # Libreria OpenCV per operazioni sulle immagini, come il ridimensionamento.
import os  # Libreria per interagire con il sistema operativo (es. gestire percorsi).
import glob  # Libreria per trovare file che corrispondono a un pattern.
from tqdm import tqdm  # Libreria per gestire visivamente barre di avanzamento.

# --- DEFINIZIONE DELLA CLASSE DATASET ---
# Eredita da 'Dataset' di PyTorch per integrarsi con i suoi strumenti, come il DataLoader.
class DatasetH5(Dataset):
    # --- METODO COSTRUTTORE (`__init__`) ---
    # Viene eseguito una sola volta all'inizio. Prepara il dataset.
    def __init__(self, path, img_size=640, clip_range=(-1500, 1500)):
        # Salva i parametri di configurazione come attributi della classe.
        self.img_size = img_size  # Dimensione finale delle immagini.
        self.clip_min, self.clip_max = clip_range  # Intervallo per il clipping dei valori dei pixel.
        self.class_id = 0  # ID di classe fisso (0), dato che abbiamo solo una classe ("regione attiva").
        
        # --- LOGICA DI CACHING ---
        # Definisce una sottocartella 'cache' dove verranno salvati i dati pre-elaborati.
        cache_dir = 'cache'
        # Crea la cartella 'cache' se non esiste gia'. 'exist_ok=True' evita errori se la cartella esiste.
        os.makedirs(cache_dir, exist_ok=True)
        
        # Costruisce un nome univoco per i file di cache basato sul nome della cartella dei dati (es. "Train" o "Validation").
        cache_name = os.path.basename(os.path.normpath(path))
        # Crea il percorso completo per il file di cache delle etichette (es. 'cache/Train_labels.npy').
        label_cache = os.path.join(cache_dir, f'{cache_name}_labels.npy')
        # Crea il percorso completo per il file di cache delle dimensioni originali delle immagini.
        shape_cache = os.path.join(cache_dir, f'{cache_name}_shapes.npy')

        # Cerca tutti i file .h5 nel percorso dato, li ordina e ne salva la lista.
        self.h5_files = sorted(glob.glob(os.path.join(path, '*.h5')))
        # Salva il numero totale di file trovati.
        self.n = len(self.h5_files)

        # Controlla se entrambi i file di cache esistono gia'.
        if os.path.exists(label_cache) and os.path.exists(shape_cache):
            # --- CARICAMENTO VELOCE DA CACHE (AVVVII SUCCESSIVI) ---
            print(f"Caricamento rapido da cache per '{cache_name}'...")
            # Carica l'array delle etichette dal file .npy. 'allow_pickle=True' e' necessario perche' le etichette sono in una lista di array.
            self.labels = np.load(label_cache, allow_pickle=True)
            # Carica l'array delle dimensioni dal file .npy.
            self.shapes = np.load(shape_cache)
            print(f"Cache caricata per {len(self.labels)} file. Avvio del training...")
        else:
            # --- CREAZIONE DELLA CACHE (PRIMO AVVIO LENTO) ---
            print(f"Cache non trovata. Creazione della cache per '{cache_name}' (lento solo la prima volta)...")
            
            # Inizializza le liste che conterranno i dati estratti.
            self.labels = []
            self.shapes = []
            bad_labels_count = 0  # Contatore per le etichette scartate.
            
            # Itera su ogni file .h5 trovato, mostrando una barra di avanzamento.
            for h5_path in tqdm(self.h5_files, desc=f"Caching metadata from {path}"):
                try:  # Blocco per gestire errori di lettura dei singoli file.
                    # Apre il file .h5 in modalita' lettura. 'with' assicura la chiusura automatica.
                    with h5py.File(h5_path, 'r') as f:
                        # Estrae il dataset del magnetogramma.
                        magnetogram_data = f['magnetogram/data']
                        # Legge le dimensioni originali (altezza, larghezza).
                        orig_h, orig_w = magnetogram_data.shape
                        # Aggiunge le dimensioni alla lista 'self.shapes'.
                        self.shapes.append([orig_h, orig_w])

                        # Accede al gruppo dei metadati HARP.
                        harp_group = f['harp/metadata']
                        image_labels = []  # Lista temporanea per le etichette di questa immagine.
                        
                        # Itera su ogni regione attiva trovata nei metadati.
                        for harp_id in harp_group:
                            # Estrae gli attributi della regione attiva corrente.
                            harp_attrs = harp_group[harp_id].attrs
                            
                            # Definisce le chiavi necessarie per calcolare una bounding box.
                            required_keys = ['CRPIX1', 'CRPIX2', 'CRSIZE1', 'CRSIZE2']
                            # Controlla se tutti gli attributi necessari sono presenti.
                            if not all(key in harp_attrs for key in required_keys):
                                continue  # Se ne manca uno, salta questa regione.

                            # Converte le dimensioni in numeri decimali.
                            w_abs = float(harp_attrs['CRSIZE1'])
                            h_abs = float(harp_attrs['CRSIZE2'])

                            # Controlla che le dimensioni siano positive.
                            if w_abs <= 0 or h_abs <= 0:
                                bad_labels_count += 1
                                continue  # Se non lo sono, scarta l'etichetta.

                            # Calcola le coordinate del centro e le dimensioni, normalizzandole rispetto alle dimensioni dell'immagine (sommando meta' della larghezza/altezza alla coordinata di origine CRPIX).
                            x_center_norm = float(harp_attrs['CRPIX1']) / orig_w
                            y_center_norm = float(harp_attrs['CRPIX2']) / orig_h
                            width_norm = w_abs / orig_w
                            height_norm = h_abs / orig_h

                            # Controlla che il centro della bounding box sia dentro l'immagine.
                            if not (0.0 < x_center_norm < 1.0 and 0.0 < y_center_norm < 1.0):
                                bad_labels_count += 1
                                continue  # Se e' fuori, scarta l'etichetta.
                            
                            # Aggiunge l'etichetta valida (formato YOLO) alla lista temporanea.
                            image_labels.append([self.class_id, x_center_norm, y_center_norm, width_norm, height_norm])
                        
                        # Aggiunge le etichette di questa immagine alla lista principale.
                        self.labels.append(np.array(image_labels, dtype=np.float32) if image_labels else np.empty((0, 5), dtype=np.float32))

                except Exception as e:  # Se si verifica un errore grave durante la lettura.
                    print(f"Errore grave durante la lettura del file {h5_path}: {e}")
                    # Aggiunge placeholder per mantenere l'allineamento degli indici.
                    self.labels.append(np.empty((0, 5), dtype=np.float32))
                    self.shapes.append([0, 0])

            # Stampa un riepilogo delle etichette scartate, se ce ne sono.
            if bad_labels_count > 0:
                print(f"ATTENZIONE: Trovate e scartate {bad_labels_count} etichette corrotte.")

            # Converte la lista di liste 'self.shapes' in un unico array NumPy.
            self.shapes = np.array(self.shapes, dtype=np.float64)
            
            # SALVA I DATI PROCESSATI NELLA CACHE PER USO FUTURO.
            print(f"Salvataggio della cache in '{path}'...") # NOTA: Stampa il percorso dei dati, non della cache
            np.save(label_cache, self.labels)  # Salva le etichette.
            np.save(shape_cache, self.shapes)  # Salva le dimensioni.
            print("Cache creata. I prossimi avvii saranno istantanei.")

    # --- METODO `__len__` ---
    # Restituisce il numero totale di campioni nel dataset.
    def __len__(self):
        return self.n  # Restituisce il numero di file contati all'inizio.

    # --- METODO `__getitem__` ---
    # Carica e restituisce un singolo campione (immagine + etichetta) dato un indice.
    def __getitem__(self, index):
        # Ottiene percorso e etichette pre-caricate per l'indice richiesto.
        h5_path = self.h5_files[index]
        labels_tensor = torch.from_numpy(self.labels[index])
        
        try:  # Blocco per gestire errori di apertura file (es. file corrotti).
            # Tenta di aprire il file H5 e leggere i dati dell'immagine.
            with h5py.File(h5_path, 'r') as f:
                data = f['magnetogram/data'][:]  # Carica l'intero array in memoria.

            # Pulisce i dati da eventuali valori non numerici (NaN/inf).
            if np.isnan(data).any() or np.isinf(data).any():
                data = np.nan_to_num(data, nan=0.0, posinf=0.0, neginf=0.0)
            
            # Pre-processa l'immagine: clipping, normalizzazione e ridimensionamento.
            clipped_data = np.clip(data, self.clip_min, self.clip_max)
            normalized_data = (clipped_data - self.clip_min) / (self.clip_max - self.clip_min)
            resized_image = cv2.resize(normalized_data, (self.img_size, self.img_size), interpolation=cv2.INTER_LINEAR)
            
            # Converte a 3 canali (duplicando il canale unico) per compatibilita' con YOLOv7.
            image_rgb = np.stack([resized_image] * 3, axis=-1)
            # Converte l'array NumPy in un tensore PyTorch e riordina le dimensioni in [C, H, W].
            image_tensor = torch.from_numpy(image_rgb.transpose(2, 0, 1)).float()
            
            # Restituisce il campione completo.
            return image_tensor, labels_tensor, h5_path, self.shapes[index]

        except Exception as e:  # Se si verifica un qualsiasi errore durante il caricamento.
            # Stampa un avviso e ignora il dato corrotto.
            print(f"\nATTENZIONE: Ignorato file corrotto o illeggibile: {os.path.basename(h5_path)}")
            
            # Restituisce un'immagine nera per non interrompere il training.
            placeholder_image = torch.zeros((3, self.img_size, self.img_size))
            return placeholder_image, labels_tensor, h5_path, self.shapes[index]
\end{lstlisting}

\subsection{Integrazione del Data Loader nella Logica di Training}
La creazione di un data loader personalizzato è un passo necessario, ma non sufficiente per il completo riadattamento del modello, la logica appena descritta necessita di essere integrata nel flusso di lavoro. Per questo motivo, la decisione più adeguata è stata quella di intervenire direttamente sul codice sorgente di YOLOv7, in particolare su quegli script che sono responsabili dell'addestramento e della validazione del modello.

\subsubsection{Adattamento della Logica di Training}
La creazione del nuovo Data Loader ha richiesto alcune modifiche allo script \texttt{train.py}, che rappresenta il motore del processo di training. L'obiettivo principale era quello di permettere al modello di scegliere dinamicamente quale loader utilizzare -quello di default per immagini standard o quello personalizzato per i dati HDF5- senza alterare la logica del ciclo di addestramento. Per ottenere ciò, sono state apportate le seguenti modifiche:
\begin{itemize}
    \item \textbf{Importazione condizionale} È stata importata la classe \texttt{DatasetH5} dallo script \texttt{utils/dataset\_h5.py}, in modo da renderla disponibile al momento della creazione del dataset.
    \begin{lstlisting}[language=Python, caption={[Importazione condizionale in train.py] Importazione della classe \texttt{DatasetH5} personalizzata.}, label={lst:import_dataset_h5}]
# Importa la classe personalizzata per la gestione dei dataset in formato HDF5.
from utils.dataset_h5 import DatasetH5
\end{lstlisting}
\item \textbf{Implementazione di una Funzione collate Personalizzata} \\
    L'implementazione della funzione \texttt{h5\_collate\_fn} è stata necessaria affinchè l'assemblaggio dei dati in batch avvenisse correttamente. Quando il Data Loader raggruppa più campioni per formare un batch, il modello deve essere in grado di effettuare un'associazione univoca tra ogni set di etichette (ovvero le coordinate delle bounding box) ed i relativi magnetogrammi di origine. Il compito principale della funzione è proprio quello di aggiungere a ciascuna etichetta un indice numerico che corrisponde alla posizione dell'immagine all'interno del batch; così facendo, il formato delle etichette è trasformato per includere tale indicatore, risolvendo ogni ambiguità. Questo ordinamento è necessario per permettere al ciclo di addestramento di confrontare le predizioni del modello con le etichette reali per ogni singolo magnetogramma, garantendo il corretto funzionamento dell'intero processo di apprendimento.
    \begin{lstlisting}[language=Python, caption={[Funzione h5\_collate\_h5 in train.py] Implementazione di una Funzione collate Personalizzata h5\_collate\_h5 in train.py}, label={lst:import_dataset_h5}]
def h5_collate_fn(batch):
    imgs, labels, paths, shapes = zip(*batch)
    batched_labels = []
    for i, label in enumerate(labels):
        if label.shape[0] > 0:
            batch_idx = torch.full((label.shape[0], 1), i, device=imgs[0].device)
            label_with_batch_idx = torch.cat((batch_idx, label.to(imgs[0].device)), 1)
            batched_labels.append(label_with_batch_idx)
    if len(batched_labels) > 0:
        targets = torch.cat(batched_labels, 0)
    else:
        targets = torch.empty(0, 6, device=imgs[0].device)
    return torch.stack(imgs, 0), targets, paths, shapes
logger = logging.getLogger(__name__)
\end{lstlisting}
    \item \textbf{Attivazione tramite file di configurazione} \\
    Per integrare il nuovo Data Loader, piuttosto che utilizzare un argomento da riga di comando, è stato implementato un meccanismo di \textit{attivazione contestuale} basato sul file di configurazione del dataset. All'avvio, lo script ispeziona il file \texttt{.yaml} fornito tramite argomento \texttt{-{}-data} alla ricerca della chiave booleana \texttt{is\_h5}: se è impostata su \texttt{True}, il framework capisce che deve gestire un dataset in formato \texttt{.h5} e attiva automaticamente il dataset personalizzato
    \begin{lstlisting}[language=Python, caption={[Attivazione DataLoader in train.py] Attivazione tramite file di configurazione nello script train.py}, label={lst:activate_dataset_h5}]
# Apre e legge il file di configurazione del dataset (es. harp.yaml)
with open(opt.data) as f:
    data_dict = yaml.load(f, Loader=yaml.SafeLoader)

# Controlla se la chiave 'is_h5' esiste e ha valore True; altrimenti, imposta False
is_h5_dataset = data_dict.get('is_h5', False)
\end{lstlisting}
    \item \textbf{Iniezione della logica di caricamento dati} Il cambiamento più significativo è avvenuto nella sezione in cui è creato l'oggetto \texttt{dataset}, grazie all'inserimento di una struttura condizionale \texttt{if-else} che controlla il valore dell'opzione \texttt{-{}-h5}: 
    \begin{itemize}
        \item se attivata, lo script ignora il loader di default \texttt{LoadImagesAndLabels} e istanzia la classe \texttt{DatasetH5}, passandole i parametri necessari (es. percorso ai dati, dimensione di immagini); 
        \item se disattivata, lo script maniene il suo comportamento originale.
    \end{itemize}
    \begin{lstlisting}[language=Python, caption={[Iniezione della logica di caricamento dati in train.py] Logica condizionale per la selezione dinamica del data loader in \texttt{train.py}.}, label={lst:dataloader_choice}]
# Controlla se l'opzione '--h5' e' stata attivata dalla riga di comando.
if opt.h5:
    # Se opt.h5 e' True, crea un'istanza del data loader personalizzato per i file HDF5.
    dataset = DatasetH5(
        train_path,      # Passa il percorso alla cartella contenente i file .h5.
        img_size=imgsz   # Passa la dimensione a cui verranno ridimensionate le immagini.
    )
# Altrimenti, se l'opzione '--h5' non e' stata attivata...
else:
    # ...esegue il codice originale, creando un'istanza del data loader di default di YOLOv7.
    dataset = LoadImagesAndLabels(
        train_path,          
        imgsz,               
        batch_size,          
        augment=True,        
        hyp=hyp,             
        rect=opt.rect,       
        cache_images=opt.cache_images, 
        single_cls=opt.single_cls,    
        stride=int(stride),  
        pad=0.0,             
        image_weights=opt.image_weights, 
        prefix=colorstr('train: ')     
    )
\end{lstlisting}
\end{itemize}
L'approccio appena descritto ed adottato per lo scopo è noto come \textit{feature flagging}: si tratta di una tecnica di sviluppo software che permette -all'occorrenza- di attivare/disattivare funzionalità specifiche di un'applicazione senza dover modificare il codice sorgente per intero, ma solo in modo mirato. In questo modo, è stato possibile estendere la funzionalità di YOLOv7 in modo modulare, rimanendo coerente al funzionamento originale e garantendo che la nuova logica sia eseguita solo quando esplicitamente richiesto.

\subsubsection{Estensione della Logica di Validation}
Per mantenere una coerenza all'interno del modello e per garantire che non ci siano conflitti in fase di validazione, è stato necessario apportare modifiche anche allo script \texttt{test.py}. Un modello addestrato deve necessariamente essere validato utilizzando lo stesso processo di caricamento e pre-elaborazione. Per questo motivo, sono state effettuate alcune modifiche simili rispetto a quelle apportate in \texttt{train.py}:
\begin{itemize}
       \item \textbf{Importazione condizionale}\\
       È stata aggiunta l'istruzione per importare la classe \texttt{DatasetH5} dallo script \texttt{utils/dataset\_h5.py}, in modo da renderla disponibile nello script di validazione, assieme alla funzione \texttt{degault\_collate} di default di PyTorch per assemblare i batch ed importata per essere utilizzata nella funzione \texttt{h5\_collate\_fn}.
    \begin{lstlisting}[language=Python, caption={[Importazione condizionale in test.py] Importazione della classe \texttt{DatasetH5} personalizzata.}, label={lst:import_dataset_h5_2}]
# Importa la classe personalizzata per la gestione dei dataset in formato HDF5.
from utils.dataset_h5 import DatasetH5

# Importa la funzione di default di PyTorch per l'assemblaggio dei batch.
from torch.utils.data.dataloader import default_collate
\end{lstlisting}
    \item \textbf{Implementazione di una Funzione collate Personalizzata} Analogamente allo script di training, è stata aggiunta una funzione \texttt{h5\_collate\_fn} semplificata, con lo scopo di raggruppare i singoli campioni caricati dalla classe \texttt{DatasetH5} in un unico batch. D'altronde, durante la fase di training, tale funzione doveva necessariamente aggiungere l'indice del batch ad ogni etichetta per il calcolo della loss; durante la fase di validazione, tale passaggio non è più necessario. Pertanto, per mantenere l'efficienza, la funzione si occupa semplicemente di impilare i tensori in modo standard.
    \begin{lstlisting}[language=Python, caption={[Funzione h5\_collate\_h5 in test.py] Implementazione di una Funzione collate Personalizzata h5\_collate\_h5 in test.py}, label={lst:h5_arg}]
def h5_collate_fn(batch):
    """Funzione custom per raggruppare i dati provenienti da DatasetH5 in un batch."""
    return default_collate(batch)
\end{lstlisting}
    \item \textbf{Aggiunta di un Parametro}\\
    La funzione principale \texttt{test} è stata estesa per accettare un nuovo argomento booleano \texttt{is_magnetogram}. Lo scopo è di fungere da "flag" interno. Esso è passato dallo script \texttt{train.py} durante la validazione, a fine epoca, per informare la funzione \texttt{test} che sta per ricevere dati di tipo magnetogramma piuttosto che immagini standard. Il meccanismo è il seguente: non si attiva il caricamento dei dati H5, bensì controlla i comportamenti successivi (ad esempio la normalizzazione delle immagini e la visualizzazione dei risultati), assicurando che vengano trattati come dati scientifici.
    \begin{lstlisting}
        [language=Python, caption={[Aggiunta del Parametro in test.py] Aggiunta del parametro is\_magnetogram in test.py}, label={lst:h5_arg}]
        def test(data,
         ...,
         is_magnetogram=False):
    \end{lstlisting}
    \item \textbf{Creazione Dinamica del Data Loader} \\
    Con il seguente blocco, si vuole rendere lo script flessibile, in grado sia di validare il dataset standard che quello H5 senza modifiche manuali. Legge il file di configurazione \texttt{.yaml} del dataset e cerca la chiave \texttt{is\_h5}: se è \texttt{True}, istanzia la classe \texttt{DatasetH5} e crea un Data Loader di PyTorch che la utilizza assieme alla \texttt{h5\_collate\_fn} semplificata; se \texttt{False} o non presente, lo script esegue la funzione originale \texttt{create\_dataloader}, mantenendo a pieno la compatibilità con dataset di immagini tradizionali.
    \begin{lstlisting}[language=Python, caption={[Creazione Dinamica del Data Loader in test.py] Introduzione di una logica condizionale per scegliere dinamicamente quale data loader utilizzare in test.py}, label={lst:h5_arg}]
    is_h5_dataset = data.get('is_h5', False)
        if is_h5_dataset:
            print("Utilizzo del DataLoader custom per dataset .h5")
            dataset = DatasetH5(path=data[task], img_size=imgsz)
            dataloader = torch.utils.data.DataLoader(dataset,
                                                    batch_size=batch_size,
                                                    shuffle=False,
                                                    num_workers=8, # Puoi usare opt.workers se disponibile
                                                    pin_memory=True,
                                                    collate_fn=h5_collate_fn)
        else:
            # Logica originale di YOLOv7
            dataloader = create_dataloader(data[task], imgsz, batch_size, gs, opt, pad=0.5, rect=True,
                                        prefix=colorstr(f'{task}: '))[0]
\end{lstlisting}
\item \textbf{Normalizzazione Condizionale delle Immagini} \\
Per evitare di processare i dati erroneamente, la normalizzazione dei valori dei pixel è applicata in modo selettivo, così da non ricadere in una "doppia normalizzazione". Sappiamo che le immagini standard hanno valori di pixel nell'intervallo [0,255], mentre i magnetogrammi sono normalizzati già nella classe \texttt{DatasetH5} (tramite clipping e scalatura min-max): è per questo che la flag \texttt{is\_magnetogram} agisce, proprio per assicurare che divisione per 255.0 sia saltata quando si tratta di dati scientifici, preservando la corretta scala dei valori.
\begin{lstlisting}
    [language=Python, caption={[Normalizzazione Condizionale delle Immagini in test.py] Aggiunta del controllo su is\_magnetogram in test.py}, label={lst:h5_arg}]
    if not is_magnetogram:
            img /= 255.0 
\end{lstlisting}
\item \textbf{Adattamento delle Funzioni di Visualizzazione} \\
Infine, per garantire che i risultati della validazione siano visivamente corretti ed interpretabili, sono state modificate le chiamate alle funzioni di plotting.
\begin{lstlisting}
    [language=Python, caption={[Adattamento delle Funzioni di Visualizzazione in test.py] Modifica della chiamata alla funzione di plotting in test.py}, label={lst:h5_arg}]
    Thread(target=plot_images, args=(img, targets, paths, f, names), kwargs={'is_magnetogram': is_magnetogram}, daemon=True).start()
\end{lstlisting}
\end{itemize}

\subsubsection{Personalizzazione delle Utility di Visualizzazione}
L'ultimo passo dell'adattamendo del framework è stata inerente alla visualizzazione corretta dei risultati. Le funzioni di plotting standard di YOLOv7, contenute nel modulo \texttt{utils/plots.py}, sono state progettate per operare su immagini a tre canali (BGR). Tuttavia, i magnetogrammi sono immagini a singolo canale, in cui ogni pixel ha un valore di intensità. Applicare direttamente la funzione originale a questi dati avrebbe comportato una visualizzazione non corretta dal punto di vista cromatico, o anche un errore. Per risolvere tale problema, è stata modificata la funzione \texttt{plot\_images}, introducendo una \textit{logica condizionale} che ispeziona le dimensioni del tensore dell'immagine in input:
\begin{itemize}
    \item \textbf{Caso standard - 3 canali (BGR)} In presenza di un'immagine a 3 canali, lo script mantiene il comportamento originale, effettuando la conversione da BGR a RGB.
    \item \textbf{Caso personalizzato - 1 canale} In presenza di un'immagine a singolo canale, la logica è la seguente:
    \begin{itemize}
        \item Rimuove la dimensione del canale mediante comando \texttt{squeeze()}, trasformando il tensore in una matrice 2D.
        \item Utilizza la libreria \texttt{matplotlib} per visualizzare la matrice come un'immagine in scala di grigi (\texttt{cmap='gray'}).
    \end{itemize}
\end{itemize}
\begin{lstlisting}[language=Python, caption={[Personalizzazione delle Utility di Visualizzazione in plots.py] Logica condizionale per la visualizzazione di immagini a singolo canale (scientifiche) e a tre canali (standard).}, label={lst:plot_logic}]
# Controlla se la prima dimensione del tensore dell'immagine e' 1 (ovvero, se ha un solo canale).
if img.shape[0] == 1:
    # --- Blocco per immagini a singolo canale ---
    
    # Converte l'array NumPy dell'immagine in un tensore di PyTorch.
    im = torch.from_numpy(img)
    
    # Disegna l'immagine usando matplotlib:
    # im.squeeze(0) rimuove la dimensione del canale (da [1, H, W] a [H, W]), necessaria per imshow.
    # cmap='gray' imposta la mappa di colori in scala di grigi, corretta per dati scientifici.
    plt.imshow(im.squeeze(0), cmap='gray')

# Altrimenti, se l'immagine ha piu' di un canale (si assume 3 canali).
else:
    # Esegue la funzione originale di YOLOv7
    img = img[..., ::-1].transpose(2, 0, 1)
    img = np.ascontiguousarray(img)
    im = torch.from_numpy(img)
\end{lstlisting}

\section{Tracking Multi-Oggetto con Norfair}
Solo dopo aver terminato l'adattamento di YOLOv7, è stato possibile sviluppare un'applicazione in grado di utilizzare il suddetto modello addestrato per eseguire un'analisi temporale. Prendendo come base lo script dimostrativo ufficiale fornito da Norfair per l'integrazione con YOLOv7, è stato sviluppato lo script \texttt{track.py}, con una logica implementativa che si adatta con coerenza allo scopo. L'intera sezione, quindi, descrive la struttura di tale script per il tracciamento multi-oggetto su una sequenza di magnetogrammi.

    \chapter{Sperimentazione e Analisi dei Risultati}
\label{cap:4}
\section{Preparazione del Dataset}
La fase sperimentale del presente lavoro di tesi si è basata sul dataset SDO/HMI relativo al Ciclo Solare 24. Una delle sfide principali è stata la gestione dell'enorme mole di dati scientifici grezzi e la loro predisposizione in un formato compatibile con le risorse computazionali a disposizione.

\subsection{Analisi Dimensionale del Dataset Originale}
Il dataset di partenza, costituito da magnetogrammi scientifici in formato HDF5 (\texttt{.h5}), presentava dimensioni proibitive per un addestramento diretto in assenza di infrastrutture di calcolo ad alte prestazioni. Infatti, la ripartizione originale dei dati prevedeva:
\begin{itemize}
    \item \textbf{Training Set}: 1,8 TB (corrispondenti a 81.436 magnetogrammi);
    \item \textbf{Validation Set}: 351 GB (corrispondenti a 16.012 magnetogrammi);
    \item \textbf{Test Set}: 356 GB (corrispondenti a 16.155 magnetogrammi);
\end{itemize}
Complessivamente, il volume dei dati ammontava a 2,5 TB, per un totale di 113.603 file. L'analisi preliminare ha evidenziato un peso medio per singolo magnetogramma di circa 22,07 MB, dovuto alla natura del file che conserva valori fisici del campo magnetico in virgola mobile ad alta precisione, oltre a numerosi metadati strumentali.

\subsection{Dataset Sperimentale}
In linea con la metodologia operativa definita nel Capitolo \ref{cap:3} (Sezione \ref{sec:approccio_alternativo}), per l'esecuzione degli esperimenti non sono stati utilizzati i dati grezzi, bensì la loro versione convertita e ottimizzata per il framework YOLOv7 originale, senza alcuna modifica.\\
L'applicazione della pipeline di pre-elaborazione all'intero archivio ha prodotto un dataset di dimensione totale di 4,75 GB (con un rapporto di compressione 500:1), che ha reso possibile l'esecuzione dei test su hardware a singola GPU, abbattendo i tempi di I/O dati dalla navigazione della struttura interna del file scientifico.

\section{Configurazione degli Esperimenti}
\label{sec:configurazione_esperimenti}
Per valutare le prestazioni del modello YOLOv7 e l'impatto della quantità di dati su di esse, la sperimentazione è stata suddivisa in due configurazioni distinte, entrambe basate sul dataset convertito:
\begin{itemize}
    \item \textbf{Subset Ridotto - 1 GB}: Addestramento pilota su una porzione ridotta del dataset, al fine di validare rapidamente la convergenza degli iperparametri e la correttezza della pipeline.
    \item \textbf{Dataset Completo - 4,75 GB}: Addestramento definitivo sul dataset intero preparato dalla dottoranda Elizabeth Doria Rosales, volto a massimizzare la capacità di generalizzazione del modello.
\end{itemize}
Per entrambe le configurazioni, in seguito alla fase di training, sono stati generati due \textit{checkpoint} del modello:
\begin{itemize}
    \item \textbf{Best}: pesi che hanno ottenuto le migliori metriche sul validation set durante il training.
    \item \textbf{Last}: pesi al termine dell'ultima epoca di addestramento.
\end{itemize}

\section{Risultati del Rilevamento con YOLOv7}
In questa sezione, vengono riportate le metriche quantitative ottenute, per entrambe le configurazioni esplorate nella Sezione \ref{sec:configurazione_esperimenti}. 

\subsection{Risultati del Subset}
L'addestramento sul Subset ridotto ha rappresentato la \textbf{prima fase sperimentale}, fondamentale per comprendere il comportamento della rete in condizione di scarsità di dati. I risultati ottenuti sono quelli emblematici delle sfide legate all'addestramento delle \textit{Deep Neural Network}: il modello ha mostrato un'ottima capacità iniziale di apprendimento, seguita però da un rapido e marcato degrado delle prestazioni dovuto al fenomeno dell'\textit{overfitting}.

\subsubsection{Training}
L'analisi dell'andamento dell'addestramento è visibile nei grafici complessivi generati durante le epoche. Come si osserva nella Figura \ref{fig:results_1gb} riportata di seguito, il modello raggiunge il suo picco prestazionale attorno all'epoca 50. In questa fase, le metriche di validazione (in particolare mAP e Recall) trovano i valori massimi, indicando che la rete sta imparando correttamente le caratteristiche morfologiche delle regioni attive.\\
Tuttavia, proseguendo l'addestramento oltre questo punto, si nota una \textbf{divergenza critica}: mentre il modello continua a minimizzare la \textit{loss} sui dati di training (segno che sta memorizzando gli esempi), le prestazioni sui dati di validazione iniziano a scendere costantemente invece di stabilizzarsi. Questo è il segnale inequivocabile che il dataset da 1 GB non possiede una varianza sufficiente per sostenere un addestramento lungo senza incorrere in overfitting.
\begin{figure}[H]
    \centering
    \includegraphics[width=1\textwidth]{Figs/Cap4/results_1gb.png} 
    \caption[Curve di addestramento e validazione (Subset 1 GB)]{Curve di addestramento e validazione per il subset da 1 GB. È evidente il picco di prestazioni (mAP@0.5 $\approx$ 0.6) attorno all'epoca 50, seguito da un lento ma inesorabile degrado delle metriche di validazione.}
    \label{fig:results_1gb}
\end{figure}

\subsubsection{Test (Checkpoint Best)}
Il \textbf{\textit{checkpoint} Best} rappresenta lo stato del modello salvato dal sistema in presenza della massima performance sul validation set (corrispondente al picco discusso sopra). In questa fase "ideale", nonostante la limitata quantità di dati, il modello era riuscito a generalizzare correttamente.\\
I test effettuati su questo checkpoint hanno prodotto risultati molto incoraggianti, con una mAP@0.5 di 0.604, Precision e Recall che dimostrano che il modello possiede un buon bilanciamento, riuscendo ad identificare la maggior parte delle strutture rilevanti con una buona confidenza. Il tutto è mostrato in Figura \ref{fig:pr_best_1gb}.\\
Questo risultato dimostra che l'architettura di YOLOv7 è idonea al compito scientifico, ma necessita di essere fermata al momento giusto (\textit{Early Stopping}) se i dati sono pochi.
\begin{figure}[H]
    \centering
    \includegraphics[width=0.7\textwidth]{Figs/Cap4/pr_curve_best_1gb.png}
    \caption[Curva Precision-Recall (Checkpoint Best, Subset 1 GB)]{Curva Precision-Recall relativa ai pesi migliori (Checkpoint Best). L'area sottesa alla curva (mAP@0.5) raggiunge il valore di 0.604, indicando una buona capacità di rilevamento prima dell'insorgere dell'overfitting.}
    \label{fig:pr_best_1gb}
\end{figure}

\subsubsection{Test (Checkpoint Last)}
Il \textbf{\textit{checkpoint} Last} corrisponde al modello finale, salvato al termine di tutte le epoche di addestramento previste. Qui, le conseguenze dell'overfitting diventano misurabili e severe. A causa dell'eccessiva specializzazione sul training set, il modello ha perso la capacità di riconoscere le regioni attive mai viste prima.\\
Il crollo delle prestazioni, come mostrato in Figura \ref{fig:pr_last_1gb}, è drastico rispetto alla configurazione Best: mAP@0.5 scende a 0.224 (rispetto a 0.605 del Best).
\begin{figure}[H]
    \centering
    \includegraphics[width=0.7\textwidth]{Figs/Cap4/pr_curve_last_1gb.png}
    \caption[Curva Precision-Recall (Checkpoint Last, Subset 1 GB)]{Curva Precision-Recall del modello finale (Checkpoint Last) del Subset Ridotto. Il valore di mAP@0.5 crolla a 0.224, confermando che il prolungamento dell'addestramento su un dataset ridotto ha danneggiato le capacità di generalizzazione della rete.}
    \label{fig:pr_last_1gb}
\end{figure}
Dal punto di vista visivo, questo degrado si traduce in una "cecità" verso le regioni più piccole.\\
Come mostrato nel confronto seguente (Figura \ref{fig:visual_compare_1gb}), mentre il \textit{Ground Truth} evidenzia numerose regioni attive di varia intensità, il modello finale è diventato estremamente conservativo: ignora la quasi totalità delle regioni attive (\textbf{Falsi Negativi}), rilevando sporadicamente solo quelle più estese ed evidenti.
% --- PRIMA PARTE (Pagina 1) ---
\begin{figure}[H]
    \centering
    % Immagine A
    \includegraphics[width=1\textwidth]{Figs/Cap4/visual_labels_1gb.jpg}
    \vspace{0.1cm}
    {\footnotesize \textbf{(a) Ground Truth}}
    % Didascalia della prima parte 
    \caption[Confronto qualitativo (1 GB)]{Prima parte del Confronto Qualitativo.}
    \label{fig:visual_compare_1gb}
\end{figure}
\clearpage % FORZA IL SALTO PAGINA 
% --- SECONDA PARTE (Pagina 2) ---
\begin{figure}[H]
    \ContinuedFloat % Mantiene lo stesso numero di figura
    \centering
    % Immagine B
    \includegraphics[width=1\textwidth]{Figs/Cap4/visual_pred_1gb.jpg}
    \vspace{0.1cm}
    {\footnotesize \textbf{(b) Predizione Modello Last}}
    % Didascalia completa
    \caption[]{Seconda parte del Confronto Qualitativo. In alto (a) nella pagina precedente le annotazioni reali, qui sopra (b) le predizioni. Si nota l'elevato numero di Falsi Negativi.}
\end{figure}

\subsection{Risultati del Dataset Completo}
L'addestramento sul dataset completo ha confermato l'ipotesi principale: l'aumento della varietà e quantità dei dati ha agito come fattore di regolarizzazione naturale, mitigando notevolmente il fenomeno dell'overfitting riscontrato nell'esperimento precedente.

\subsubsection{Training}
Il grafico dell'andamento dell'addestramento mostra un comportamento radicalmente diverso rispetto al caso del subset ridotto. Come visibile nella Figura \ref{fig:results_full}, le curve di validazione non presentano più la caratteristica forma a "picco e crollo". Al contrario, si osserva una \textbf{crescita progressiva delle metriche} (mAP, Precision, Recall) che tende a stabilizzarsi nella seconda metà della fase di training. Questo indica che il modello non sta più memorizzando i singoli esempi, ma sta apprendendo caratteristiche generalizzabili che rimangono valide anche sui dati di validazione. 
\begin{figure}[H]
    \centering
    \includegraphics[width=1\textwidth]{Figs/Cap4/results_full.png} 
    \caption[Curve di addestramento e validazione (Dataset Completo)]{Curve di addestramento sul Dataset Completo. A differenza del caso precedente, si nota una notevole stabilità: le metriche di validazione crescono e si mantengono costanti senza degradare, dimostrando che il modello ha raggiunto una convergenza robusta.}
    \label{fig:results_full}
\end{figure}

\subsubsection{Test (Checkpoint Best)}
Il \textbf{\textit{checkpoint} Best}, relativo al dataset completo, ha fatto registrare prestazioni eccellenti, raggiungendo un valore di mAP@0.5 pari a 0.597. Questo risultato è particolarmente significativo se confrontato con l'esperimento precedente: sebbene il valore numerico assoluto sia simile a quello del subset (0.604), la solidità statistica è ben diversa. Ottenere tale precisione su un dataset più vasto e variegato conferma che la rete neurale, quando alimentata con una quantità adeguata di informazioni, riesce ampiamente a generalizzare le caratteristiche morfologiche delle regioni attive senza ricorrere alla memorizzazione. La curva Precision-Recall, mostrata nella Figura \ref{fig:pr_best_full}, evidenzia un'area sottesa molto ampia, indice di un classificatore robusto.
\begin{figure}[H]
    \centering
    \includegraphics[width=0.65\textwidth]{Figs/Cap4/pr_curve_best_full.png}
    \caption[Curva Precision-Recall (Checkpoint Best, Dataset Completo)]{Curva Precision-Recall relativa ai pesi migliori (Checkpoint Best) del Dataset Completo. Il mAP raggiunge il valore di 0.597, confermando l'alta capacità di apprendimento del modello in presenza di un dataset adeguatamente dimensionato.}
    \label{fig:pr_best_full}
\end{figure}

\subsubsection{Test (Checkpoint Last)}
L'analisi del \textbf{\textit{checkpoint} Last}, salvato al termine del ciclo di addestramento, offre la conferma definitiva della bontà dell'approccio basato sul dataset completo. Da quanto si evince nella Figura \ref{fig:pr_last_full}, il modello finale ha raggiunto una mAP@0.5 di 0.385. Sebbene si registri una flessione fisiologica tra il picco assoluto e lo stato finale, il modello mantiene una capacità predittiva solida. Questo dimostra che l'aumento dei dati ha agito efficacemente, prevenendo il crollo delle prestazioni (\textit{catastrophic forgetting}) che si era verificato nel primo esperimento.
\begin{figure}[H]
    \centering
    \includegraphics[width=0.65\textwidth]{Figs/Cap4/pr_curve_last_full.png}
    \caption[Curva Precision-Recall (Checkpoint Last, Dataset Completo)]{Curva Precision-Recall del modello finale (Checkpoint Last) del Dataset Completo. Il valore di mAP@0.5 si assesta a 0.385, dimostrando una tenuta delle prestazioni decisamente superiore rispetto al crollo osservato nel dataset ridotto.}
    \label{fig:pr_last_full}
\end{figure}
Anche l'analisi qualitativa conferma questi dati numerici. Come visibile nel confronto seguente (Figura \ref{fig:visual_compare_full}), le predizioni del modello riescono a localizzare correttamente la maggior parte delle regioni attive, mantenendo una buona copertura spaziale anche alla fine dell'addestramento e riducendo drasticamente i Falsi Negativi rispetto al modello addestrato su pochi dati.
% --- PRIMA PARTE (Pagina 1) ---
\begin{figure}[H]
    \centering
    % Immagine A
    \includegraphics[width=1\textwidth]{Figs/Cap4/visual_labels_last_full.jpg}
    \vspace{0.1cm}
    {\footnotesize \textbf{(a) Ground Truth}}
    % Didascalia della prima parte 
    \caption[Confronto qualitativo (Dataset Completo)]{Prima parte del Confronto Qualitativo.}
    \label{fig:visual_compare_full}
\end{figure}
\clearpage % FORZA IL SALTO PAGINA 
% --- SECONDA PARTE (Pagina 2) ---
\begin{figure}[H]
    \ContinuedFloat % Mantiene lo stesso numero di figura
    \centering
    % Immagine B
    \includegraphics[width=1\textwidth]{Figs/Cap4/visual_pred_last_full.jpg}
    \vspace{0.1cm}
    {\footnotesize \textbf{(b) Predizione Modello Last}}
    % Didascalia completa
    \caption[]{Seconda parte del Confronto Qualitativo. In alto (a) le annotazioni reali (Ground Truth), in basso (b) le predizioni. Il modello dimostra di aver appreso le caratteristiche morfologiche, rilevando le regioni attive principali senza i numerosi errori di omissione riscontrati nell'esperimento da 1 GB.}
\end{figure}

\subsection{Confronto e Discussione}
\label{sec:confronto_discussione}
L'analisi congiunta dei due esperimenti condotti permette di trarre conclusioni significative riguardo l'applicabilità delle reti neurali convoluzionali profonde (come YOLOv7) nell'ambito della \textit{Space Weather}. Il confronto tra l'addestramento sul Subset Ridotto e quello sul Dataset Completo evidenzia come la quantità dei dati non influenzi solo la performance finale, ma la stabilità stessa del processo di apprendimento.

\subsubsection{Confronto delle Metriche}
La tabella \ref{tab:confronto_metriche} riassume le prestazioni registrate nelle due configurazioni.
\begin{table}[H]
\centering
\begin{tabular}{lccc}
\hline
Configurazione & Best (mAP) & Last (mAP) & Delta ($\Delta$) \\
\hline
Subset (1 GB) & 0.604 & 0.224 & -63\% \\
\textbf{Dataset Completo (4.75 GB)} & 0.597 & \textbf{0.385} & \textbf{-35\%} \\
\hline
\end{tabular}
\caption[Confronto metriche prestazionali]{Confronto sintetico delle prestazioni tra le due configurazioni sperimentali. Si evidenzia come il dataset completo riduca drasticamente il calo di prestazioni.}
\label{tab:confronto_metriche}
\end{table}
Dai dati emerge un \textbf{paradosso apparente}: il modello addestrato su pochi dati ha raggiunto un picco numerico leggermente superiore (0.604 contro 0.597), ma si tratta di un risultato ingannevole. Nel caso del subset ridotto, il picco è stato raggiunto grazie alla memorizzazione delle caratteristiche specifiche del training set, incapace però di sostenere la generalizzazione nel lungo termine. Al contrario, il dataset completo ha prodotto un modello estremamente più robusto: il divario contenuto tra il Best e il Last indica che la rete ha efficacemente appreso le \textit{feature} morfologiche delle regioni attive, mantenendo la capacità di riconoscerle anche al termine di un lungo addestramento.

\subsubsection{Stabilità e Generalizzazione}
La \textbf{differenza più critica} risiede nel comportamento visivo e nella tenuta del modello. Il test sul subset da 1 GB ha mostrato i sintomi classici di un apprendimento instabile: man mano che l'addestramento procedeva oltre il punto di ottimo, la rete dimenticava come riconoscere le strutture più piccole e meno evidenti, focalizzandosi solo su pochi esempi macroscopici. Il test sul dataset completo, invece, ha mantenuto una sensibilità elevata (Recall) durante tutto il processo. L'aumento della varietà dei dati ha agito come una forma di regolarizzazione implicita, impedendo ai pesi della rete di specializzarsi eccessivamente e garantendo una copertura uniforme sia sulle grandi regioni attive che sulle più piccole.\\
In conclusione, la sperimentazione dimostra che l'architettura YOLOv7 è \textbf{idonea al rilevamento delle regioni attive} sui magnetogrammi SDO/HMI, a patto che venga alimentata con una mole di dati pre-convertiti sufficiente a rappresentare la varianza statistica del fenomeno solare.

\section{Validazione del Tracking Multi-Oggetto}
Conclusa l'analisi delle prestazioni della detection statica, l'ultima fase sperimentale si concentra sulla validazione della consistenza temporale delle predizioni. Per questa analisi, la pipeline di tracciamento è stata applicata utilizzando i checkpoint risultanti da YOLOv7 addestrato sul \textbf{dataset completo (4,75 GB)} (poiché rivelatosi quello con risultati migliori). L'obiettivo specifico è quantificare la capacità del sistema di mantenere stabile l'identità delle regioni attive attraverso frame successivi, confrontando direttamente le prestazioni ottenute con la configurazione del checkpoint Best rispetto a quelle ottenute con il checkpoint Last.

\subsection{Metodologia di Valutazione}
Per quantificare oggettivamente le qualità del tracciamento, le metriche prese in considerazione sono state quelle standard per il \textit{Multi-Object Tracking} (MOT):
\begin{itemize}
    \item \textbf{MOTA (Multiple Object Tracking Accuracy)}: metrica globale che combina falsi positivi, falsi negativi e scambi di identità (\textit{ID Switches}); rappresenta, quindi, l'accuratezza complessiva del sistema.
    \item \textbf{IDF1 (Identification F1 Score)}: misura la capacità del sistema di attribuire e mantenere lo stesso ID corretto ad un oggetto per tutta la durata; rappresenta, quindi, il parametro più critico per la coerenza temporale.
    \item \textbf{ID Switches}: misura il numero totale di volte in cui il sistema commette un errore cambiando l'ID ad un oggetto che è rimasto lo stesso.
\end{itemize}

\section{Risultati del Tracking}
In questa sezione, sono presentati i risultati quantitativi e qualitativi, evidenziando come la qualità del modello di rilevamento influenzi direttamente la stabilità del tracciamento.

\subsection{Analisi Quantitativa}
La Tabella \ref{tab:tracking_metrics} riassume le prestazioni misurate sui dati di test per le due configurazioni dei checkpoint.
\begin{table}[H]
\centering
\begin{tabular}{lccc}
\hline
\textbf{Configurazione} & \textbf{MOTA} ($\uparrow$) & \textbf{IDF1} ($\uparrow$) & \textbf{ID Switches} ($\downarrow$) \\
\hline
\textbf{Best} & \textbf{40.8\%} & \textbf{66.5\%} & 628 \\
Last & 24.0\% & 47.6\% & \textbf{384} \\
\hline
\end{tabular}
\vspace{0.2cm}
\caption[Metriche di Tracciamento (Dataset Completo)]{Confronto delle prestazioni di tracciamento. I pesi Best garantiscono un'accuratezza globale (MOTA) e una stabilità di identità (IDF1) nettamente superiori.}
\label{tab:tracking_metrics}
\end{table}

\subsubsection{Discussione dei Risultati}
Dal confronto emerge la \textbf{netta superiorità della configurazione Best}. Il valore di IDF1 (66.5\%) conferma che il sistema, nel suo punto di ottimo, riesce a mantenere l'identità corretta delle regioni attive per la maggior parte della loro durata, un requisito fondamentale per le analisi scientifiche. Il calo del MOTA nella configurazione Last (dal 40.8\% al 24.0\%) è coerente con la diminuzione delle capacità di rilevamento discussa nella Sezione \ref{sec:confronto_discussione} (mAP inferiore).\\
Un'\textbf{osservazione particolare} merita il dato degli ID Switches, che appare inferiore nel modello Last (384 contro 628). Tale risultato, tuttavia, non deve essere interpretato come una maggiore stabilità, bensì come un artefatto dovuto al basso valore di Recall (circa 30\%). Il modello finale, omettendo il rilevamento di numerose regioni attive (generando quindi falsi negativi), riduce statisticamente le occasioni in cui l'algoritmo può commettere un errore di scambio, risultando in un numero assoluto di switch ingannevolmente basso.

\subsection{Analisi Qualitativa}
Per valutare il comportamento del sistema in scenari reali di diversa complessità, l'analisi visiva è stata suddivisa in \textbf{tre casistiche rappresentative}.

\subsubsection{Primo Scenario: Alta Densità di Regioni }
L'analisi dello scenario ad alta attività evidenzia in modo netto l'importanza della strategia di selezione del modello (\textit{checkpointing}) basata sulla validation loss. Di fronte ad un disco solare densamente popolato da regioni attive eterogenee (risalente al 16 maggio 2012), le due configurazioni offrono \textbf{prestazioni drasticamente diverse}.\\
Come mostrato nella Figura \ref{fig:scenario1_best}, il modello con i pesi migliori dimostra un'\textbf{elevata sensibilità} (Recall). La rete riesce a tracciare la quasi totalità delle regioni attive presenti, sovrapponendo correttamente i propri bounding box (blu) alle annotazioni di Ground Truth (rosso) su tutto il disco. Il sistema gestisce efficacemente la complessità della scena, rilevando con precisione sia le vaste strutture centrali sia le regioni minori situate in prossimità del bordo.\\
Al contrario, la Figura \ref{fig:scenario1_last} illustra il comportamento del modello all'ultima epoca di addestramento sullo stesso frame temporale. In questa configurazione si osserva un \textbf{degrado delle prestazioni}: il modello traccia solamente un numero esiguo di regioni (circa 4 su oltre una decina presenti), ignorando completamente la maggior parte delle regioni attive. Questo elevato tasso di \textit{missed detections} suggerisce che, nelle fasi finali del training, la rete abbia subito un'instabilità o una perdita di capacità di generalizzazione (\textit{catastrophic forgetting}). \\
Il confronto visivo tra le due Figure (\ref{fig:scenario1_best} e \ref{fig:scenario1_last}) convalida la scelta operativa di utilizzare il \textbf{\textit{checkpoint} Best}, l'unico in grado di garantire una copertura del segnale adeguata per il tracking delle regioni attive solari.
\begin{figure}[H]
    \centering
    \includegraphics[width=0.6\textwidth]{Figs/Cap4/best_scenario1.jpg}
    \caption[Scenario 1 (Modello Best)]{Visualizzazione delle prestazioni in uno scenario ad alta densità di regioni attive con il Checkpoint Best. I bounding box blu (predizioni con ID Norfair) si sovrappongono efficacemente alla quasi totalità dei box rossi (Ground Truth con ID HARP).}
    \label{fig:scenario1_best}
\end{figure}
\begin{figure}[H]
    \centering
    \includegraphics[width=0.6\textwidth]{Figs/Cap4/last_scenario1.jpg}
    \caption[Scenario 1 (Modello Last)]{Visualizzazione dello stesso frame temporale elaborato con il Checkpoint Last. Si osserva un drastico calo delle performance: il modello ignora la maggior parte delle regioni attive presenti (box rossi privi di corrispettivo blu).}
    \label{fig:scenario1_last}
\end{figure}

\subsubsection{Secondo Scenario: Bassa Densità di Regioni Attive}
Il secondo scenario di test valuta le prestazioni in condizioni di attività solare ridotta, selezionando un magnetogramma (risalente al 6 giugno 2019) caratterizzato da un disco solare prevalentemente quieto, con la presenza di sole tre regioni attive di dimensioni modeste e ben distanziate.\\
La Figura \ref{fig:scenario2_best} mostra i risultati ottenuti con il checkpoint best. La rete conferma la sua \textbf{eccellente affidabilità}, tracciando correttamente tutte e tre le regioni attive presenti. I bounding box si sovrappongono con precisione alle annotazioni di Ground Truth, dimostrando che il modello è in grado di risolvere anche segnali di intensità minori o dimensioni ridotte, senza generare falsi positivi nelle aree vuote.\\
Al contrario, la Figura \ref{fig:scenario2_last} evidenzia il comportamento del modello last. Anche in questo contesto semplificato, privo di affollamento, si osserva un \textbf{fallimento} nel rilevare due delle tre regioni presenti, limitandosi a tracciare solo la struttura più evidente. Le altre due regioni sono completamente ignorate, generando falsi negativi. Questo risultato è determinante: attesta che il checkpoint finale soffre di una \textbf{generalizzata perdita di sensibilità} (recall), rendendolo inadatto al monitoraggio operativo indipendentemente dalla complessità dell'attività solare del momento.
\begin{figure}[H]
    \centering
    \includegraphics[width=0.6\textwidth]{Figs/Cap4/best_scenario2.jpg} 
    \caption[Scenario 2 (Modello Best)]{Visualizzazione in regime di bassa attività solare con Checkpoint Best. Il modello dimostra una perfetta capacità di copertura: tutte le tre regioni attive presenti (box rossi) vengono correttamente identificate e tracciate dal sistema (box blu), confermando l'alta affidabilità del checkpoint selezionato anche su oggetti di piccole dimensioni.}
    \label{fig:scenario2_best}
\end{figure}
\begin{figure}[H]
    \centering
    \includegraphics[width=0.6\textwidth]{Figs/Cap4/last_scenario2.jpg}
    \caption[Scenario 2 (Modello Last)]{Confronto sullo stesso frame a bassa attività con Checkpoint Last. Si evidenzia nuovamente l'instabilità dei pesi finali: il modello riesce a rilevare solo la regione più grande (HARP 7366), mancando completamente le due regioni adiacenti più piccole. Ciò conferma che il modello Last ha perso la capacità di generalizzare su segnali più deboli.}
    \label{fig:scenario2_last}
\end{figure}

\subsubsection{Terzo Scenario: Coerenza Temporale (Tracking Sequence)}
    \chapter{Conclusioni e Sviluppi Futuri}
\label{cap:5}
Il presente capitolo conclude il lavoro di tesi, ripercorrendo le tappe fondamentali del percorso di ricerca ed analizzando i risultati ottenuti. Sono discussi sia i punti di forza che le limitazioni della metodologia proposta. Infine, sono proposte le linee guida per i possibili sviluppi futuri volti a potenziare l'architettura e le capacità predittive del sistema.

\section{Sintesi del Lavoro Svolto}
Il percorso di ricerca ha affrontato le sfide tipiche dell'applicazione del \textit{Deep Learning} a dati scientifici. In una prima fase, è stata definita una metodologia per la costruzione di un dataset rappresentativo, selezionando campioni significativi del Ciclo Solare 24 e gestendo la complessità del formato nativo HDF5 e dei metadati HARP.\\
Successivamente, sono stati esplorati due approcci complementari per l'addestramento della rete neurale:
\begin{itemize}
    \item \textbf{Approccio Ingegneristico \textit{Proof of Concept}}, focalizzato sull'adattamento del Data Loader di YOLOv7 per l'ingestione diretta dei dati scientifici;
    \item \textbf{Approccio Operativo}, basato sulla pre-conversione e normalizzazione del dataset, che ha permesso di condurre una sperimentazione estensiva e di validare le performance del modello. 
\end{itemize}
Infine, l'integrazione del modulo di tracciamento multi-oggetto ha esteso le capacità del sistema dalla semplice localizzazione statica alla ricostruzione dinamica delle traiettorie, permettendo di preservare l'identità delle regioni attive nonostante la rotazione solare. Il sistema sviluppato, quindi, unendo le capacità di generalizzazione di YOLOv7 alla coerenza temporale di Norfair, pone le basi per la realizzazione di strumenti di monitoraggio in tempo reale sempre più affidabili.

\section{Discussione dei Risultati}
L'analisi sperimentale condotta nel Capitolo \ref{cap:4} ha fornito evidenze quantitative e qualitative rilevanti, delineando sia le potenzialità che i limiti dell'approccio proposto.

\subsection{Efficacia del Deep Learning su Dati Solari}
I risultati ottenuti con il dataset pre-convertito completo confermano che YOLOv7 è \textbf{idoneo al rilevamento delle regioni attive}, raggiugendo una precisione media (mAP@0.5) di circa 0.60 nella configurazione ottimale. Il sistema ha dimostrato di saper generalizzare correttamente le caratteristiche morfologiche delle regioni attive, distinguendo efficacemente il segnale dal background, cioè la fotosfera quieta.\\
Un risultato cruciale emerge dal confronto tra l'addestramento su dataset ridotto (1 GB) e quello completo (4.75 GB): l'aumento della mole di dati ha agito come potente regolarizzatore, rendendo \textbf{stabile il processo di apprendimento} e migliorando drasticamente la capacità del modello finale di riconoscere le regioni anche al termine del training. Questi risultati dimostrano che l'integrazione tra le moderne tecniche di \textit{Deep Learning} ed i dati scientifici rappresenta una strada percorribile e promettente.

\subsection{Criticità della Selezione del Modello: Best vs Last}
Uno dei contributi più significativi di questa tesi risiede nell'analisi comparativa tra i \textit{Checkpoint Best} (validati durante il training) e \textit{Last} (stato finale della rete). La sperimentazione ha evidenziato un fenomeno marcato di \textbf{perdita di sensibilità} (\textit{catastrophic forgetting}) nelle fasi finali dell'addestramento, particolarmente evidente nel modello Last. Mentre il modello Best garantisce un'elevata Recall, rilevando sia le grandi strutture centrali che le piccole regioni periferiche, il modello finale tende a diventare eccessivamente conservativo, ignorando le regioni minori. \\
Questa osservazione ha un'\textbf{implicazione operativa fondamentale}: in un sistema di monitoraggio reale, non è sufficiente affidarsi alla convergenza finale dell'addestramento, ma è imperativo implementare strategie di \textit{Early Stopping} basate su metriche di validazione rigorose.

\subsection{Robustezza del Tracciamento}
L'integrazione con Norfair ha dimostrato che il tracciamento temporale è un compito robusto, parzialmente disaccoppiato dalla qualità del rilevamento. Anche quando il modello di detection perde sensibilità (come nel caso Last), il sistema di tracking riesce a \textbf{mantenere la coerenza dell'identità} delle regioni attive individuate. Tuttavia, per massimizzare l'affidabilità scientifica e l'indice IDF1, è necessario alimentare il tracker con un rilevatore ad alta sensibilità (modello Best), garantendo continuità temporale anche per le regioni vicine al limbo solare.

\section{Sviluppi Futuri}
La sperimentazione effettuata, assieme ai relativi risultati, costituiscono una base solida per ulteriori ricerche. Alla luce delle esperienze maturate e delle tecnologie sviluppate, si identificano le seguenti linee di evoluzione prioritarie.

\subsection{Consolidamento dell'Architettura End-to-End}
Come discusso nel Capitolo \ref{cap:3} (Sezioni \ref{sec:adattamento_yolo}), è stato sviluppato un \textit{Proof of Concept} (PoC) per permettere a YOLOv7 di leggere nativamente i file HDF5, bypassando la fase di pre-conversione. Sebbene la sperimentazione principale si sia basata su dati convertiti per motivi di efficienza computazionale, il futuro del progetto risiede nel consolidamento di questa architettura \textit{end-to-end}.

\subsection{Estensione Temporale del Dataset}
Attualmente, il dataset copre porzioni significative del Ciclo Solare 24. Un'evoluzione necessaria prevede l'\textbf{estensione dell'addestramento} all'intero archivio storico di SDO (oltre un decennio di dati), includendo anche il corrente Ciclo 25. Disporre di un dataset che abbracci più cicli solari permetterebbe di catturare la naturale variabilità tra i diversi periodi di attività. Questo garantirebbe una visione d'insieme più completa e assicurerebbe che il modello sia in grado di generalizzare correttamente, senza limitarsi alle caratteristiche specifiche di un singolo ciclo.

\subsection{Forecasting}
Il passo logico successivo al rilevamento e tracciamento è la \textbf{previsione} (\textit{forecasting}). Avendo a disposizione le traiettorie storiche e l'evoluzione morfologica estratte dal sistema proposto, è possibile addestrare modelli predittivi (ad esempio reti ricorrenti LSTM o Transformer) per anticipare l'evoluzione delle regioni attive e la possibile insorgenza di brillamenti solari (solar flares). Tali strumenti predittivi potranno in futuro supportare gli operatori nella previsione degli eventi avversi, contribuendo attivamente alla mitigazione dei rischi associati allo Space Weather e alla protezione delle infrastrutture tecnologiche critiche.
    \bibliographystyle{apalike}
    
    
    %\bibliographystyle{plainnat} % use this to have URLs listed in References
    %\cleardoublepage
    %\cleardoublepage
    \pagestyle{plain} % Imposta lo stile della pagina su plain per la bibliografia
    \bibliography{References/references} 
    % \pagestyle{empty}% Path to your References.bib file
    %\chapter*{Acknowledgments}
\label{cap:6}
Si ringrazia Elizabeth Doria Rosales, dottoranda presso Università degli Studi di Trento / Università della Calabria, per il contributo fornito in veste di co-supervisore esterno, per la selezione dei dati, il background teorico e pratico sulle proprietà fisiche dei sistemi trattati e per avermi offerto la possibilità di aver potuto assisterla ad una parte della sua attività di ricerca.

    \newpage
    \thispagestyle{empty} % Rende la pagina vuota (senza numerazione)
    \mbox{} % Crea una pagina vuota

    %\newpage
    %\thispagestyle{empty} % Rende la pagina vuota (senza numerazione)
    %\mbox{} % Crea una pagina vuota
\end{document}