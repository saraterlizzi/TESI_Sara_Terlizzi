\newpage
\thispagestyle{empty}
\null
%\newpage
%\thispagestyle{empty}
%\null
%\newpage
%\thispagestyle{empty}
%\null
%\newpage
\selectlanguage{english}
\begin{abstract}
Negli ultimi anni, l'Intelligenza Artificiale ha acquisito un ruolo sempre più centrale nell'ambito dell'Eliofisica per il monitoraggio dello Space Weather. Partendo dalla necessità di identificare le regione attive solari, sorgenti dei principali fenomeni energetici, si è voluto sviluppare un sistema basato su Deep Learning in grado di rilevare e seguire l'evoluzione di tali strutture; il tutto è stato realizzato utilizzando i magnetogrammi rilevati dal satellite SDO e integrando l'architettura di YOLOv7 con la libreria di tracciamento Norfair. Il flusso di lavoro realizzato unisce sotto un'unica pipeline la localizzazione delle regioni attive e la ricostruzione temporale della loro traiettoria temporale. Gli obiettivi principali del progetto sono: (I) Elaborare i dati scientifici HARP per costruire un dataset annotato idoneo all'apprendimento supervisionato, (II) Addestrare il framework YOLOv7 con i dati scientifici, implementando sia moduli di caricamento dedicati (Proof of Concept) sia procedure di pre-elaborazione dei dati scientifici stessi, (III) Integrare il sistema con la libreria Norfair per garantire il tracciamento temporale degli oggetti identificati, e infine (IV) Verificare sperimentalmente l'efficacia del sistema, analizzando le prestazioni di rilevamento (detection) e la stabilità del tracciamento (tracking).

\begin{center}
    \textbf{Abstract (English)}
\end{center}
In recent years, Artificial Intelligence has acquired an increasingly central role in the field of Heliophysics for Space Weather monitoring. Starting from the necessity to identify solar active regions, sources of the main energetic phenomena, the aim was to develop a Deep Learning-based system capable of detecting and following the evolution of sush structures; this was achieved using magnetograms detected by the SDO satellite and integrating the YOLOv7 architecture with the Norfair tracking library. The implemented workflow unites, under a single pipeline, the localization of active regions and the temporal reconstructions of their trajectory. The main objectives of the project are: (I) Process HARP scientific data to build an annoted dataset suitable for supervised learing, (II) Train the YOLOv7 framework with scientific data, implementing both dedicated loading modules (Proof of Concept) and pre-processing procedures for the scientific data itself, (III) Integrate the system with the Norfair library to ensure the temporal tracking of identified objects, and finally (IV) Experimentally verify the effectiveness of the system, analyzing detection performance and tracking stability.

\end{abstract}

\selectlanguage{italian}