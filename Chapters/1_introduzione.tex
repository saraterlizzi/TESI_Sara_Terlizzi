\chapter{Introduzione}
\label{cap:1}
Nel presente capitolo, si andrà a spiegare il contesto scientifico in cui si inserisce il lavoro di tesi, illustrando le motivazioni alla base della scelta del tema affrontato e definendo l'obiettivo principale della ricerca. In particolare, l'attenzione è rivolta allo studio delle regioni attive solari, in quanto sedi privilegiate dei fenomeni energetici più intensi osservati sul Sole.\\
Il capitolo fornisce inoltre una panoramica delle implicazioni scientifiche e applicative legate all'analisi di tali regioni e anticipa la struttura generale della tesi.\\

\section{Contesto e Motivazione}
Il Sole è da sempre oggetto di numerosi studi nell'ambito dell'astrofisica, in quanto sorgente di fenomeni magnetici ed energetici che influenzano l'ambiente spaziale circostante. Tra tali fenomeni, i \textbf{flare solari} rappresentano eventi improvvisi ed altamente energetici, caratterizzati dal rilascio di grandi quantità di energia sottoforma di radiazione elettromagnetica. \\
Numerose osservazioni hanno evidenziato come tali eventi abbiano origine principalmente nelle \textbf{regioni attive solari}, aree localizzate della fotosfera caratterizzate da intensi e complessi campi magnetici, spesso associate alla presenza di macchie solari. La complessità strutturale e l'evoluzione temporale delle regioni attive svolgono un ruolo fondamentale nell'origine dei flare solari e dei fenomeni ad essi associati, quali le espulsioni di massa coronale (\textit{Coronal Mass Ejections, CME}) e le particelle energetiche solari (\textit{Solar Energetic Particles, SEP}).\\
Quando tali eventi si propagano verso la Terra, possono dare origine a \textbf{tempeste geomagnetiche} e a una vasta gamma di effetti sull'ambiente spaziale circumterrestre, con conseguenze negative sulle infrastrutture tecnologiche \cite{camporeale2018machine}. Tra le manifestazioni più rilevanti vi è la generazione di \textbf{correnti geomagneticamente indotte} (\textit{Geomagnetically Induced Currents, GIC}) nelle reti elettriche terrestri, che possono causare danni ai trasformatori e interruzioni della fornitura di energia, come dimostrato da eventi storici quali il \textit{blackout} della rete elettrica di Hydro-Québec del marzo 1989 \cite{bolduc2002gic}, durante il quale l'intero sistema passò da condizioni operative nominali allo spegnimento completo in circa 92 secondi, lasciando oltre sei milioni di persone senza elettricità in una giornata particolarmente fredda. Ulteriori esempi sono rappresentati dai numerosi guasti ai trasformatori che ridussero drasticamente la capacità della rete elettrica sudafricana durante le intense tempeste geomagnetiche dell'ottobre 2003 \cite{gaunt2007transformer}.\\
Le emissioni in raggi X e nell'ultravioletto estremo (\textit{Extreme Ultraviolet, EUV}) prodotte dai flare solari più intensi possono modificare la struttura dell'ionosfera sul lato diurno della Terra, compromettendo la propagazione dei segnali radio e influenzando il funzionamento dei sistemi globali di navigazione satellitare (\textit{Global Navigation Satellite Systems, GNSS}) \cite{hapgood2010ionospheric}. Tali sistemi, oltre alla navigazione, forniscono servizi di temporizzazione ad alta precisione, fondamentali per settori critici come quello finanziario. Ulteriori effetti includono un aumento del \textit{drag} atmosferico sui satelliti in orbita bassa, che può determinare una progressiva perdita di quota e, nei casi più estremi, il rientro prematuro in atmosfera. Un esempio recente di questo fenomeno è rappresentato dalla perdita di una parte della flotta di satelliti Starlink in seguito a un episodio di intensa attività geomagnetica.\\
Sia le espulsioni di massa coronale sia i flare solari sono inoltre associati a \textbf{tempeste di radiazione solare}, ovvero improvvisi aumenti del flusso di particelle energetiche (protoni, particelle alfa e ioni più pesanti), che possono determinare un significativo incremento dell'ambiente radiativo nello spazio circumterrestre e, in alcuni casi, anche all'interno dell'atmosfera terrestre, fino a raggiungere il livello del mare. Le tempeste di radiazione hanno effetti su una vasta gamma di sistemi elettrici ed elettronici e possono rappresentare un rischio radiativo, seppur limitato, per l'uomo, sia nello spazio sia a bordo di aeromobili. Tali effetti risultano particolarmente rilevanti per l'aviazione, soprattutto considerando il ruolo sempre più centrale dei dispositivi digitali nei sistemi di controllo del volo, nonché nei sistemi di navigazione e comunicazione.\\
Alla luce di ciò, l'identificazione e il tracciamento temporale delle regioni attive solari risultano di fondamentale importanza per comprendere e, potenzialmente, prevedere l'attività solare.
\begin{figure}[H]
  \centering
  \includegraphics[width=0.4\textwidth]{Figs/Cap1/flare_solare.jpg}
  \caption[Flare Solare]{Immagine del Sole in cui sono in atto flare solari}
\end{figure}

\section{Obiettivo della Tesi}
Da quanto illustrato finora, lo studio dell'attività solare riveste un ruolo fondamentale nell'analisi dello \textit{space weather}, ovvero l'insieme dei fenomeni astrofisici che possono influenzare le tecnologie spaziali e terrestri.\\
Poiché gli eventi più rilevanti in questo contesto hanno origine prevalentemente nelle regioni attive del Sole, la presente tesi si propone di:
\begin{enumerate}
    \item Sviluppare un sistema automatico per l'identificazione delle regioni attive solari attraverso l'analisi dei magnetogrammi del disco solare, prodotti dallo strumento \textbf{Helioseismic and Magnetic Imager (HMI)} a bordo del satellite \textbf{Solar Dynamics Observatory (SDO)}.
    \item Utilizzare il \textit{data product} \textbf{HARP} (\textit{Helioseismic and Magnetic Imager Active Region Patches}), fornito dal team scientifico del Solar Dynamics Observatory, che offre una rappresentazione strutturata delle aree magneticamente attive sulla superficie solare, permettendo l'annotazione dei magnetogrammi e lo sviluppo di un metodo di localizzazione automatica delle regioni attive, basato sull'estrazione delle \textit{bounding box}.
    \item Riadattare il modello di \textit{object detection} \textbf{YOLOv7}, rendendolo utilizzabile in un contesto astrofisico e integrandolo con la libreria \textbf{Norfair} per il \textit{tracking} temporale delle regioni attive.
    \item Utilizzare un dataset annotato per l’addestramento del modello YOLOv7 (opportunamente modificato), con l’obiettivo di riconoscere automaticamente le regioni attive solari.
\end{enumerate}
\textbf{YOLOv7} è un modello di \textit{object detection} preesistente, ideato per identificare e localizzare oggetti in immagini RGB in modo accurato ed efficiente \cite{wang2022yolov7}. Per adattarlo all'analisi dei magnetogrammi solari e delle regioni attive descritte nei dati HARP, sono state apportate le seguenti modifiche:
\begin{itemize}
  \item caricamento di immagini in formato scientifico (file \texttt{.h5} contenenti magnetogrammi e \textit{bounding box} HARP);
  \item estrazione delle \textit{bounding box} a partire dai dati HARP;
  \item addestramento del modello per la rilevazione automatica delle regioni attive in immagini solari;
  \item integrazione con la libreria Norfair per il \textit{tracking} temporale delle regioni attive rilevate.
\end{itemize}
Tali modifiche consentono di sfruttare al meglio YOLOv7 in un contesto scientifico diverso da quello per cui è stato originariamente concepito, ma mantenendo al contempo l'efficienza e la rapidità del modello originale.\\
I dati utilizzati in questo lavoro sono pubblicamente accessibili dal Joint Science Operations Center (JSOC), la struttura responsabile dell'elaborazione e della distribuzione dei dati del Solar Dynamics Observatory (SDO), gestita dalla Stanford University in collaborazione con la NASA \cite{pesnell2012solar}.\\
La selezione dei dati è avvenuta a cura di Elizabeth Doria Rosales, dottoranda in fisica presso l'Università di Trento.

\section{Struttura della Tesi}
La tesi è strutturata nei seguenti capitoli:
\begin{itemize}
    \item \hyperref[cap:2]{\textbf{Capitolo 2}}: Panoramica delle tecnologie e metodologie utilizzate, con focus sul modello YOLOv7 e sulla libreria Norfair.
    \item \hyperref[cap:3]{\textbf{Capitolo 3}}: Illustrazione delle soluzioni tecniche adottate.
    \item \hyperref[cap:4]{\textbf{Capitolo 4}}: Analisi dei risultati sperimentali, con valutazioni.
    \item \hyperref[cap:5]{\textbf{Capitolo 5}}: Conclusioni ed eventuali sviluppi futuri.
\end{itemize}
Infine, le conclusioni racchiudono una riflessione sui risultati ottenuti, sui limiti del lavoro effettuato e le potenziali direzioni da poter intraprendere in futuro.
